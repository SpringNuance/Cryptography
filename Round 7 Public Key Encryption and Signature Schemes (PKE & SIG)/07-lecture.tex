\documentclass[a4paper,table,dvipsnames]{article}
\usepackage{lmodern}
\usepackage[utf8]{inputenc}
\usepackage[T1]{fontenc}
\usepackage[english]{babel}
\usepackage{latexsym}
\usepackage{amsmath,amsfonts,amssymb}
\usepackage{mathpartir}
\usepackage{amsthm}
\usepackage{tikz}
\usepackage{enumitem}
\usepackage[probability,adversary,sets,operators,primitives]{cryptocode}
\usepackage{fancyhdr}
\usepackage{hyperref}
\usepackage{wrapfig}
%\usepackage[table]{xcolor}
%\usepackage[margin=2.8in]{geometry}

\pagestyle{fancy}
\usetikzlibrary{shapes,arrows,positioning}
\tikzstyle{vertex} = [draw, circle]

\theoremstyle{definition}
\newtheorem{definition}{Definition}[section]
\newtheorem{remark}{Remark}
\newtheorem{lemma}{Lemma}
\newtheorem{claim}{Claim}
\newtheorem{conjecture}{Conjecture}
\newtheorem{construction}{Construction}
\newtheorem{theorem}{Theorem}
\newtheorem{example}{Example}
\newtheorem{hint}{Hint}
\setlength{\parindent}{0ex}

\rhead[Lecture Notes 7]{Lecture Notes 7}
\lhead[CS-E4340 Cryptography, Chris Brzuska]{CS-E4340 Cryptography, Chris Brzuska}

\newcommand{\M}[1]{\ensuremath{\text{\texttt{#1}}}}
\renewcommand{\O}[1]{\ensuremath{\mathsf{#1}}}
\newcommand{\pcvar}[1]{\ensuremath{\mathit{#1}}}
%\newcommand{\pcassert}{\ensuremath{\mathbf{assert}\;}}
\newcommand{\pckw}[1]{\highlightkeyword{#1}}
\newcommand{\pctype}[1]{\mathsf{#1}}
\newcommand{\ininterface}[1]{[#1\rightarrow]} %{I^{in}_{#1}}
\newcommand{\outinterface}[1]{[\rightarrow#1]} %{I^{out}_{#1}}
\newcommand{\m}{\pcvar{m}} %{I^{out}_{#1}}
\newcommand{\ver}{\O{ver}} %{I^{out}_{#1}}
\newcommand{\kgen}{\O{kgen}}
\newcommand{\se}{\O{se}} %{I^{out}_{#1}}
\newcommand{\s}{\O{s}}
\newcommand{\pk}{\pcvar{pk}}
\newcommand{\sk}{\pcvar{sk}}



\renewcommand{\enc}{\O{enc}} %{I^{out}_{#1}}
\renewcommand{\dec}{\O{dec}} %{I^{out}_{#1}}

\renewcommand{\mac}{\O{mac}}
\renewcommand{\circ}{\rightarrow}

\newcommand\defeq{\mathrel{\overset{\makebox[0pt]{\mbox{\normalfont\tiny\sffamily
def}}}{=}}}
\def\checkmark{\tikz\fill[scale=0.4](0,.35) -- (.25,0) -- (1,.7) --
(.25,.15) -- cycle;}

\title{Lecture 7: Public-Key Encryption}
\author{Chris Brzuska}
\date{\today}

\begin{document}
See Pihla's lecture notes for the definition of public-key encryption (PKE) 
and a discussion of black-box proofs. The present lecture notes contain four section with
supplementary material.

\begin{description}
\item[Section~\ref{sec:detpke}] discusses why it's problemantic if a PKE is deterministic---it cannot be IND-CPA-secure then.
\item[Section~\ref{sec:rsaoaep}] describes why RSA-OAEP was introduced, a concrete (real-world) public-key encryption scheme.
\item[Section~\ref{sec:rom}] discusses the random oracle model.
\item[Section~\ref{sec:nonbb}] contains reading material for non-black-box constructions/proofs (cf. Pihla's discussion). 
\item[Section~\ref{sec:bb}] contains pointers to studies of black-box constructions/proofs (cf. Pihla's discussion). 
\item[Section~\ref{sec:pq}] briefly discusses post-quantum cryptography.
\end{description}

\section{Deterministic PKE}\label{sec:detpke}
If a public-key encryption (PKE) is deterministic, then the situation is
even worse than in the symmetric-key encryption case. In the symmetric-key
encryption case, the leakage of deterministic encryption means that an
observer can spot repetitions. However, in deterministic \emph{public-key} encryption,
an attacker can always check their guess for the content of a message against
the actual message. That's a much stronger capability, Therefore, randomness
is even more important than in the symmetric-key encryption case.
\paragraph{Textbook RSA} Textbook RSA usually describes a deterministic
function, also knows as a \emph{trapdoor function}. Namely, we have a
key generation algorithm $\O{tf}.\kgen$ which outputs a pair $(\pk,\sk)$
and then $f(\pk,x)$ is a one-way function, i.e., for all PPT adversaries
$\adv$, the probability that the following experiment returns $1$ should be negligible:

\begin{center}
\begin{pcvstack}
\procedure{$\O{Exp}_{\O{TDF}}(1^\lambda)$}{
(\pk,\sk)\sample\kgen(1^\lambda)\\
x\sample\bin^\lambda\\
y\gets f(\pk,x)\\
x'\sample\adv(\pk,x,1^\lambda)\\
\pcif f(\pk,x')=y:\\
\pcind\pcreturn 1\\
\pcreturn 0}
\end{pcvstack}
\end{center}


In RSA, the public-key $\pk$ consists of a modulus $N$ and an exponent $e$,
and the secret-key $\sk$ consists also of the modulus $N$ as well as 
another exponent $d$. In practice, for $e$, the value 65537 is popular due
to its binary representation since it contains only two ones, so that
calculating exponentiation is quite easy. Note that $e=3$ can be a bad
choice due to the Coppersmith attack~\url{https://en.wikipedia.org/wiki/Coppersmith%27s_attack#H%C3%A5stad%27s_broadcast_attack}

\paragraph{Trapdoor function} In cryptography, the RSA function is referred
to as a \emph{trapdoor function} or a \emph{trapdoor permutation} (since it
is bijective). Outside of cryptography, the RSA function is also referred to
as ``encryption function''. This can be quite misleading terminology, as RSA
on its own is not a full encryption scheme whereas the name seems to suggest
that RSA alone is good encryption...

\section{Encrypting with RSA}\label{sec:rsaoaep}
Encrypting with RSA is not an easy task. Firstly, one needs to add randomness
in some way (to be discussed in more detail shortly), and secondly, we need
to cope with the issue that one-wayness is an extremely weak assumption,
as we have seen in the first lecture of the course. Therefore, proving IND-CPA security merely based on a one-wayness definition seems
already quite challenging conceptually. One could go through hardcore bits~\url{http://www.wisdom.weizmann.ac.il/~oded/X/acgs.pdf},
but this seems to yield a quite inefficient encryption system.

We now first describe RSA-OAEP and then discuss the different approaches towards
proving RSA-OAEP. Before turning to RSA-OAEP, just a quick discussion about a
previous approach to randomizing RSA which failed quite strongly.

\paragraph{Padding Oracle attacks on RSAES-PKCS1-v$1_5$}
The Bleichenbacher attack on RSAES-PKCS1-v1$\_$5~\url{https://en.wikipedia.org/wiki/PKCS_1}
is a variant of the padding oracle attack which we have seen in the \emph{before the
lecture} part of lecture 4. Namely, the nice algebraic properties of RSA lend themselves
as the basis for attacks, if the
padding is not carefully done. For RSAES-PKCS1-v1$\_$5, it was shown that a partial decryption oracle (i.e., an oracle which
tells whether the padding is correct or not) can be used to recover a message encrypted 
using RSAES-PKCS1-v1$\_$5 bit by bit.

\paragraph{RSA-OAEP}
See~\url{https://en.wikipedia.org/wiki/Optimal_asymmetric_encryption_padding} for
the description of the RSA-OAEP padding scheme.



\section{Random Oracle Proofs}\label{sec:rom}
Bellare and Rogaway who proposed the OAEP~\url{https://cseweb.ucsd.edu/~mihir/papers/oaep.html} originally
carried out their proof in the \emph{random oracle model}, i.e., they assumed that the functions $H$ and $G$
are implemented by oracles which behave as public, truly random functions, so-called \emph{random oracles} (RO). While it is clear that a (deterministic,
public) hash-function
 cannot behave like a true random function and cannot be indistinguishable from
a truly random function (since, unlike a PRF, a hash-function does not have a secret key),
modeling hash-functions as random oracles is popular in the research community, since it
simplifies proofs. Namely, instead of a (very weak) one-wayness assumption (as the one we
previously described for trapdoor functions), in the proof, one can use the much stronger
assumption that $\mathsf{RO}(x)$ is uniformly random and secret as long as the attacker does not know $x$. Bellare and Rogaway
deem random oracles a solid methodology for practical analysis~\url{https://cseweb.ucsd.edu/~mihir/papers/ro.pdf},
whereas other researchers collect the flaws and problems of the methodology. Canetti, Goldreich and Halevi (CGH) were the
first to demonstrate technical and conceptual flaws in the random oracle model methodology~\url{https://eprint.iacr.org/1998/011.pdf}.
The article essentially shows that there are cryptographic schemes which can be proven
secure in the random oracle model, but are insecure whenever we replace the random oracle
by a concrete function. I.e., CGH show that \emph{there is no concrete hash-function that
makes the scheme secure}. Thereby, CGH establish that the random oracle methodology might
make us believe that certain schemes are secure, while they are blatantly insecure. The CGH
paper is a good example of how controversial the random oracle methodology is discussed in
the field: Each author wrote their own conclusion, since each of the three interpreted the
facts brought forward in the paper differently. CGH was then later criticized for providing
only contrived counterexamples, whereas, so the critics' opinion, ``natural'' cryptographic
schemes would not show such strange behaviour as the ones presented in CGH. For more recent works on understanding random
oracles, see, e.g.,~\url{https://eprint.iacr.org/2013/424} and~\url{https://eprint.iacr.org/2014/867}.

\paragraph{(Un)instantiability proofs for RSA-OAEP}
It was sought to establish that security of RSA-OAEP can \emph{only} be proven in the random oracle,
perhaps pointing out another weakness of the random oracle model. See, e.g.,~\url{https://link.springer.com/chapter/10.1007/978-3-642-01001-9_23}
for an uninstantiability result (i.e., a result showing that the random oracle might not be instantiable
by any concrete hash-function) by Kiltz and Pietrzak. However, a little later, Kiltz, O’Neill, and Smith
showed~\url{https://link.springer.com/content/pdf/10.1007/978-3-642-14623-7_16.pdf} that the hash-functions in RSA-OAEP can be securely instantiated
when making stronger assumptions on the RSA function than merely assuming that it is a one-way function.



\section{Non-black-box constructions and reductions}\label{sec:nonbb}
\paragraph{Example 1: Zero-Knowledge} In 2008, in a celebrated breakthrough result, Barak showed that treating the adversary in
a \emph{non-black-box} way might sometimes allow us to prove statements which were impossible
to prove in a black-box way~\cite{https://www.boazbarak.org/Papers/nonbb.pdf}. His idea emerged
in the context of certain zero-knowledge proofs and the difficulty of showing security against a malicious
verifier (who wants to break the zero-knowledge property). On the very high-level, the idea, here (described on page 23 of 
the article), is that if knowing the code of the verifier allows to compute a \emph{commitment} to
the code of the verifier, which later allows to show that the verifier, indeed, behaved correctly.
The details of the proof are beyond the scope of this course---but if you are curious about this
paper, then this article would be a fantastic choice for a presentation in the context of the course
MS-E1687 Advanced Topics in Cryptography~\url{https://mycourses.aalto.fi/course/view.php?id=32919}.

\paragraph{Example 2: Obfuscation} Barak's technique did not lead to a broad use of non-black-box use
of an adversary. However, non-black-box use of a cryptographic \emph{primitive} became a very popular
technique after 2013 in cryptography. Namely, in 2013, a breakthrough candidate construction by 
Garg, Gentry, Halevi, Raykova, Sahai and Waters (GGHRSW~\url{https://eprint.iacr.org/2013/451
}) convinced the cryptographic community that
\emph{obfuscation} of arbitrary circuits might be possible, after all. This was a big surprise
due to an impossibility result from 2001 by Barak, Goldreich, Impagliazzo, Rudich, Sahai, Vadhan, and 
Yang (BGIRSVY~\url{https://www.wisdom.weizmann.ac.il/~oded/p_obfuscate.html}) which showed that so-called
\emph{virtual black-box} obfuscation for general circuits is impossible. BGIRSVY also propose a very
weak notion of obfuscation known as \emph{indistinguishability obfuscation} and believed to be rather
useless, see, e.g.,~\url{https://www.wisdom.weizmann.ac.il/~oded/MC/136.html}. However, Sahai and
Waters~\url{https://eprint.iacr.org/2013/454} showed that indistinguishability obfuscation is actually 
a very useful primitive, and with the GGHRSW candidate construction together, the cryptographic
community started on fruitful and lively research on (a) applications of indistinguishability obfuscation
and (b) constructions based on reasoneable assumptions. Only this year, indistinguishability obfuscation
has been based on quite well-founded assumptions~\url{https://eprint.iacr.org/2021/1334}. You can ask
Kirthi, Estuardo or Chris if you are interested in more background on the topic of obfuscation. 

But what is the connection between obfuscation and non-black-box techniques? Obfuscation takes a circuit/program
and turns it into an non-intellegible version of that circuit/program---but it maintains the input-output
behaviour. In order to use obfuscation, we need to describe our cryptographic primitive concretely as a circuit/program.
Therefore, when relying on obfuscation, the underlying cryptographic primitives cannot be treated in a
black-box way.

\section{Black-box proofs}\label{sec:bb}
The first authors to study the power of \emph{black-box proofs} in cryptography
were Impagliazzo and Rudich who showed that one cannot build public-key encryption
from one-way permutations in black-box way~\url{https://link.springer.com/content/pdf/10.1007}.
Later, Reingold, Trevisan and Vadhan sought to cleanly characterize which techiques are captured
by black-box impossibility results \`a la Impagliazzo-Rudich and thus defined a classification
of notions of reducibility between cryptographic primitives~\url{https://omereingold.files.wordpress.com/2014/10/blackbox.pdf}.
This work was later refined by Baecher, Brzuska and Fischlin~\url{https://eprint.iacr.org/2013/101}.
Nowadays, I also think that a good definition of \emph{black-box} is to say that everything
which can be proven via this \emph{package methodology} (which Osama briefly presented) is
a black-box proof, see the Crypto Companion or  \url{https://eprint.iacr.org/2018/306}.

\section{Post-Quantum Cryptography}\label{sec:pq}
The National Institute of Standards and Technology (NIST) started a competition for
post-quantum secure public-key encryption schemes. In fact, the actual competition
concerns post-quantum secure \emph{key encapsulation mechanisms} (KEM). These are public-key
encryption schemes which only encrypt \emph{random} messages (i.e., \emph{keys}) instead
of chosen messages. Once two parties share a secret random message, they can use this random
message as a key for a symmetric encryption scheme---this is known as the KEM-DEM approach,
see, e.g., Section 4 in \url{https://eprint.iacr.org/2018/306}. I did not say much about
the NIST competition in the lecture, but if you are
interested in the current state-of-the-art of post-quantum secure
public-key encryption, then this might be a good starting point:
https://www.nist.gov/news-events/news/2020/07/nists-postquantum-cryptography-program-enters-selection-round
\end{document}