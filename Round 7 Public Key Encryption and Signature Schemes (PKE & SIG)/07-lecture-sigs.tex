\documentclass[a4paper,table,dvipsnames]{article}
\usepackage{lmodern}
\usepackage[utf8]{inputenc}
\usepackage[T1]{fontenc}
\usepackage[english]{babel}
\usepackage{latexsym}
\usepackage{amsmath,amsfonts,amssymb}
\usepackage{mathpartir}
\usepackage{amsthm}
\usepackage{tikz}
\usepackage{enumitem}
\usepackage[probability,adversary,sets,operators,primitives]{cryptocode}
\usepackage{fancyhdr}
\usepackage{hyperref}
\usepackage{wrapfig}
%\usepackage[table]{xcolor}
%\usepackage[margin=2.8in]{geometry}

\pagestyle{fancy}
\usetikzlibrary{shapes,arrows,positioning}
\tikzstyle{vertex} = [draw, circle]

\theoremstyle{definition}
\newtheorem{definition}{Definition}[section]
\newtheorem{remark}{Remark}
\newtheorem{lemma}{Lemma}
\newtheorem{claim}{Claim}
\newtheorem{conjecture}{Conjecture}
\newtheorem{construction}{Construction}
\newtheorem{theorem}{Theorem}
\newtheorem{example}{Example}
\newtheorem{hint}{Hint}
\setlength{\parindent}{0ex}

\rhead[Lecture Notes 8]{Lecture Notes 8}
\lhead[CS-E4340 Cryptography, Chris Brzuska]{CS-E4340 Cryptography, Chris Brzuska}

\newcommand{\M}[1]{\ensuremath{\text{\texttt{#1}}}}
\renewcommand{\O}[1]{\ensuremath{\mathsf{#1}}}
\newcommand{\pcvar}[1]{\ensuremath{\mathit{#1}}}
%\newcommand{\pcassert}{\ensuremath{\mathbf{assert}\;}}
\newcommand{\pckw}[1]{\highlightkeyword{#1}}
\newcommand{\pctype}[1]{\mathsf{#1}}
\newcommand{\ininterface}[1]{[#1\rightarrow]} %{I^{in}_{#1}}
\newcommand{\outinterface}[1]{[\rightarrow#1]} %{I^{out}_{#1}}
\newcommand{\m}{\pcvar{m}} %{I^{out}_{#1}}
\newcommand{\ver}{\O{ver}} %{I^{out}_{#1}}
\newcommand{\kgen}{\O{kgen}}
\newcommand{\se}{\O{se}} %{I^{out}_{#1}}
\newcommand{\s}{\O{s}}
\newcommand{\pk}{\pcvar{pk}}
\newcommand{\sk}{\pcvar{sk}}



\renewcommand{\enc}{\O{enc}} %{I^{out}_{#1}}
\renewcommand{\dec}{\O{dec}} %{I^{out}_{#1}}

\renewcommand{\mac}{\O{mac}}
\renewcommand{\sig}{\O{sig}}
\renewcommand{\circ}{\rightarrow}

\newcommand\defeq{\mathrel{\overset{\makebox[0pt]{\mbox{\normalfont\tiny\sffamily
def}}}{=}}}
\def\checkmark{\tikz\fill[scale=0.4](0,.35) -- (.25,0) -- (1,.7) --
(.25,.15) -- cycle;}

\title{Lecture 7: Signatures}
\author{Chris Brzuska}
\date{\today}

\begin{document}
\maketitle
The current lecture notes contain
\begin{itemize}
\item a discussion of applications of signature schemes
\item the syntax of signature schemes
\item security of signature schemes
\item Lamport's theorem mentioned in the lecture which establishes that one-way functions
imply one-time-secure signature scheme.
\end{itemize}

\section{Applications}
Signatures protect the authenticity and integrity of a message, i.e., if a message is signed,
then I know that this message is authorized by the owner of the signing key. For example, trusted
developers sign the code of the applications that they develop, and the operating system then
verifies that an application is trusted (i.e., the signature verifies under the key of a 
trusted developer) before allowing the application to be installed. In this case, the use
of a signature scheme is quite similar to an ordinary signature scheme, i.e., a person
explicitly authorizes a piece of text as their own---except that a digital signature is
verifiable by a machine, which is, of course, convenient. Due to this convenience, digital signatures
are used, e.g., also in e-commerce, e-governance and other trusted applications. E.g., some universities
digitally sign the diplomas which they issue.\footnote{Note that in practice, this might not necessarily
be extremely convenient, since it is not clear that the receiving university has the required
technological know-how to verify the signature.}

Moreover, signatures are an essential tool to prevent \emph{person-in-the-middle} (PITM) attacks
on key exchange protocols. Namely, a PITM attack on a key exchange protocol intends to trick to
parties to use a key which the PITM attacker knows. To be able to mount such an attack, the PITM
attacker needs to modify the messages which the two parties transmit, whence the usefulness of
using signatures which prevent such changes. Strictly speaking, signatures, of course, do not
\emph{prevent} such changes, but rather, they only make such changes detectable.

An important issue w.r.t. to signing keys is the question how to link a signing key to a person
or to a URL, since it is easy to verify that a signature is valid under a certain public-key,
but how can I be sure that it is indeed the verification key of the intended URL and not an
attacker's verification key? The way in which this is usually achieved is via \emph{certificates}.
A \emph{certificate}, interestingly, again consists of a signature! Namely, it is a signature
of a \emph{name} and a \emph{public-key} and it links these two together. However, this certificate
could, potentially, also be adversary-generated. Thus, we might also want to see a certificate for
the certificate issuer (and indeed, this is done in practice) which links the name of the issuer
to the signing key. In this way, we can create so-called \emph{certificate chains}. In the end,
however, we need a root certificate. And such root certificates are typically pre-installed in
a system. The security of the entire system hinges on the trustworthiness of the pre-installed
root certificates.

After discussing these applications of signatures, let us now turn to the definition of their
syntax and security.



\section{Syntax and Security}
Signature schemes are used for protect the authenticity and integrity of messages and are the \emph{public-key} analogue of message authentication codes (MACs). Namely, while MACs can only be verified when knowing the secret MAC key, signatures can be verified when knowing the \emph{public} verification key. I.e., a signature scheme relies on a \emph{key pair}.

\begin{definition}[Signature scheme]
    A \emph{signature} scheme $\s$ consists of three probabilistic polynomial-time (PPT) algorithms
    \begin{align*} 
      (\pk,\sk)\sample&\s.\kgen(1^\lambda) \\
      \sigma\sample&\s.\sig(\sk,x) \\
      \bin \sample &\s.\ver(\pk,x,\sigma)
    \end{align*}
    Correctness criterion: $\forall x\in\bin^*:$
      \[\probsub{(\pk,\sk)\sample\s.\kgen(1^\lambda),\sigma\sample\s.sig(\sk,x)}{\s.\ver(\pk,x,\sigma)=1}=1.\]
    i.e. when generating a key pair $(\pk,\sk)\sample\s.\kgen(1^\lambda)$ and signing $\sigma\sample\sig(\sk,x)$ for some message $x$, then $\ver(\pk,x,\sigma)=1$.
\end{definition}

\newpage
\begin{definition}[UNF-CMA for Signatures]\label{def:UNF-CMA-sig}
Let the  games $\M{Gunf\text{-}cma}^0_\s$ and $\M{Gunf\text{-}cma}^1_\s$ be defined as
\begin{center}
  \begin{pchstack}
      \pchspace
        \begin{pcvstack}
          \underline{\underline{$\M{Gunf\text{-}cma}_\s^0$}}\\
                          \\
              \procedure{Package Parameters}{
				\lambda\text{:} \< \quad \text{security parameter}\\
				%b\text{:} \< \quad \text{idealization bit}\; b=0\\
				%in\text{:} \< \quad \text{input length}\;\lambda\;\text{or}\;*\\
				%				out\text{:} \< \quad \text{output length}\,\lambda\,\text{or}\,*\\
			  \s\text{:} \< \quad \text{signature scheme} 
			              }
										\pcvspace
										              \procedure{Package State}{
																	\pk\text{:} \<\quad \text{public key}\\
																	\sk\text{:} \<\quad \text{secret key}\\
			              }
          \pcvspace
    \procedure{$\O{GETPK}()$}{
		    \pcif \pk=\bot:\\
				\pcind  (\pk,\sk)\sample\s.\kgen(1^\lambda)\\
      \pcreturn \pk}
										\pcvspace
    \procedure{$\O{SIG}(x)$}{
		  \pcif \pk=\bot:\\
			\pcind (\pk,\sk)\sample\s.\kgen(1^\lambda)\\
      \sigma\sample \s.\sig(\sk,x)\\
      \\
      \pcreturn \sigma}
   \pcvspace
		\procedure{$\O{VERIFY}(x,\sigma)$}{
		    \pcif \pk=\bot:\\
				\pcind  (\pk,\sk)\sample\s.\kgen(1^\lambda)\\
      d\gets \s.\ver(\pk,x,\sigma)\\
      \pcreturn d}
  \end{pcvstack}
              \pchspace
  \begin{pcvstack}
          \underline{\underline{$\M{Gunf\text{-}cma}_\s^1$}}\\
                                \\
              \procedure{Package Parameters}{
				\lambda\text{:} \< \quad \text{security parameter}\\
				%b\text{:} \< \quad \text{idealization bit}\; b=1\\
				%in\text{:} \< \quad \text{input length}\;\lambda\;\text{or}\;*\\
				%				out\text{:} \< \quad \text{output length}\,\lambda\,\text{or}\,*\\
			  \s\text{:} \< \quad \text{signature scheme} 
			              }
										\pcvspace
										              \procedure{Package State}{
																	\pk\text{:} \<\quad \text{public key}\\
																	\sk\text{:} \<\quad \text{secret key}\\
 				                  \mathcal{L}\text{:} \<\quad \text{list} 
			              }
          \pcvspace
    \procedure{$\O{GETPK}()$}{
		    \pcif \pk=\bot:\\
				\pcind  (\pk,\sk)\sample\s.\kgen(1^\lambda)\\
      \pcreturn \pk}
          \pcvspace
    \procedure{$\O{SIG}(x)$}{
		    \pcif \pk=\bot:\\
				\pcind  (\pk,\sk)\sample\s.\kgen(1^\lambda)\\
      \sigma\sample \s.\sig(\sk,x)\\
      \mathcal{L}\gets\mathcal{L}\cup\{(x,\sigma)\}\\
      \pcreturn \sigma}
          \pcvspace
    \procedure{$\O{VERIFY}(x,\sigma)$}{
          \pcif (x,\sigma)\in\mathcal{L}:\\
          \pcind \pcreturn 1\\
          \\
          \pcreturn 0}
  \end{pcvstack}
  \end{pchstack}
  \end{center}

A signature scheme $\s$ is UNF-CMA-secure if the real game $\M{Gunf\text{-}cma}^0_\s$ and the ideal game $\M{Gunf\text{-}cma}^1_\s$ are computationally indistinguishable, that is, for all PPT adversaries $\adv$, the advantage
\begin{align*}
&\mathbf{Adv}_{\adv}^{
  \M{Gunf\text{-}cma}^0_\s,
  \M{Gunf\text{-}cma}^1_\s}
	(\lambda) \\
:=&\abs{\prob{1=\adv\circ \M{Gunf\text{-}cma}^0_\s}-\prob{1=\adv\circ \M{Gunf\text{-}cma}^1_\s}}
\end{align*}
is negligible in $\lambda$.
\end{definition}
\paragraph{Remark.} Note that we denote $\M{Gunf\text{-}cma}^b_\s$ the unforgeability game for signatures
when $\s$ is a \emph{signature} scheme and $\M{Gunf\text{-}cma}^b_m$ the unforgeability game for MACs
when $m$ is a MAC scheme.

\section{Constructions}
In the lecture, we constructed the above signature scheme $s_{f}$ based on an injective one-way function $f$,
which signs messages of exactly length $\lambda$.
		\begin{figure}
		\centering
			\begin{pchstack}
			\begin{pcvstack}
					\procedure{$s_\text{f}.\kgen(1^\lambda)$}
				{
				\pcfor i=1..\lambda\\
				\pcind x_0^i\sample\bin^\lambda\\
				\pcind x_1^i\sample\bin^\lambda\\
				\pcind y_0^i\gets f(x_0^i)\\
				\pcind y_0^i\gets f(x_1^i)\\
				\sk\gets\begin{minipage}{0.05\linewidth}$\left(\begin{minipage}{10\linewidth}
					$x_0^1,..,x_0^\lambda$\\
					$x_1^1,..,x_1^\lambda$\end{minipage}\right)$\end{minipage}\\
				\pk\gets \begin{minipage}{0.05\linewidth}$\left(\begin{minipage}{10\linewidth} $y_0^1,..,y_0^\lambda$\\$y_1^1,..,y_1^\lambda$\end{minipage}\right)$\end{minipage}
				}\\
				\pcreturn (\sk,\pk)
			\end{pcvstack}
\pchspace
			\begin{pcvstack}
				\procedure{$s_f.\sig(\sk,m)$}
				{
					\mathsf{parse}\,\sk\text{ as}\\
				\begin{minipage}{0.041\linewidth}$\left(\begin{minipage}{10\linewidth}
					$x_0^1,..,x_0^\lambda$\\
					$x_1^1,..,x_1^\lambda$\end{minipage}\right)$\end{minipage}\\
				\pcfor i=1..\lambda\\
				\pcind z^i\gets x^i_{m[i]}\\
				\pccomment{m[i]\text{ is }i\text{th bit of }m}\\
				\sigma\gets (z^1,..,z^\lambda)\\
				\pcreturn \sigma
				}
			\end{pcvstack}
\pchspace
			\begin{pcvstack}
				\procedure{$s_{f}.\ver(\pk,m,\sigma)$}
				{					\mathsf{parse}\,\pk\text{ as}\\
				\begin{minipage}{0.041\linewidth}$\left(\begin{minipage}{10\linewidth}
					$y_0^1,..,y_0^\lambda$\\
					$y_1^1,..,y_1^\lambda$\end{minipage}\right)$\end{minipage}\\
					(z^1,..,z^\lambda)\gets \sigma\\
			\pcfor i=1..\lambda\\
				\pcind \pcif f(z^i)\neq y^i_{m[i]}:\\
				\pcind\pcind\pcreturn 0\\
				\pcreturn 1
				}
			\end{pcvstack}
			\end{pchstack}
\caption{Lamport's one-time signature scheme $s_{f}$ for messages of length $\lambda$.\label{fig:lamport}}
\end{figure}

We then stated the following theorem (The proof is one of the exercises on Exercise Sheet 6):

\begin{theorem}[Signature scheme from OWF]
If $f$ is a one-way function, then the signature scheme $s_{f}$ is a one-time secure
signature scheme for messages of length $\lambda$, i.e., for all PPT adversaries $\adv$ that make only a single $\O{SIG}$
query of length $\ell$, the advantage $\mathbf{Adv}_{\adv}^{
  \M{Gpunf\text{-}cma}^0_\s,
  \M{Gpunf\text{-}cma}^1_\s}
	(\lambda)$ is negligible.
\end{theorem}


\end{document}