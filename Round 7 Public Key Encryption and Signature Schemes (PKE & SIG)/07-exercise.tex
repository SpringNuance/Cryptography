\newif\iffull\fullfalse% do not change flags here
\newif\ifdraft\draftfalse% do not change flags here

\documentclass[envcountsame,runningheads,notitlepage]{../llncs}
\pagestyle{plain}

\usepackage[letterpaper,hmargin=3.5cm,vmargin=3cm]{geometry}
\usepackage[T1]{fontenc}
\usepackage{lmodern}
\usepackage{xifthen}
\usepackage{amsmath}
\usepackage{amsfonts}
\usepackage{amssymb}
\usepackage{lscape}
%\usepackage{amsthm}
\usepackage{xspace}
\usepackage{enumerate}
\usepackage{tikz}
\usepackage{hyperref}
\usepackage{color,soul}
\usepackage{tikz}
\usetikzlibrary{shapes,arrows, calc,fit}
\usepackage{graphicx}
\usepackage{array}
\usepackage{subcaption}
\usepackage[probability,adversary,sets,notions,operators,complexity,keys,primitives,asymptotics,advantage]{cryptocode}
\usepackage{cleveref}

\usepackage{savesym}
\usepackage{relsize}
\savesymbol{cb}
\usepackage{combelow}
\usepackage[normalem]{ulem}
\usepackage{wrapfig}
\newtheorem{crossed}{Esitehtävä}
\usepackage{enumitem}

\let\proof\relax
\let\endproof\relax
\usepackage{amsthm}

\theoremstyle{definition}
\newtheorem{graded}[crossed]{Exercise}
\newtheorem{challenge}[crossed]{Challenge exercise}

\tikzstyle{package} = [draw,rounded corners=2mm,minimum height=.7cm,minimum width=2cm]

% Colors
\newcommand\maybecolor[1]{\color{#1}}
\definecolor{lightblue}{rgb}{0.2,0.2,1.0}
\definecolor{dkred}{rgb}{0.5, 0.0,0.0}
\definecolor{lightgrey}{rgb}{0.8,0.8,0.8}
\definecolor{dkblue}{rgb}{0,0.1,0.5}
\definecolor{dkgreen}{rgb}{0,0.4,0}
\definecolor{dkred}{rgb}{0.6,0,0}
\definecolor{linkColor}{rgb}{0,0,0.5}
\definecolor{lightgray}{rgb}{.9,.9,.9}
\definecolor{darkgray}{rgb}{.4,.4,.4}
\definecolor{purple}{rgb}{0.65, 0.12, 0.82}

\ifdraft
\newcommand{\fullcolor}{\color{blue}}
\else
\newcommand{\fullcolor}{\color{black}}
\fi
\newcommand{\blk}{\color{black}}
\newcommand{\dkg}{\color{dkgreen}}

% Variable and procedures names
\renewcommand{\O}[1]{\ensuremath{\mathsf{#1}}}
\newcommand{\V}[1]{\ensuremath{\mathit{#1}}}
\newcommand{\M}[1]{\ensuremath{\text{\tt#1}}}
\def \G {\mathcal{G}}
\newcommand{\hLen}{n}




%\newcommand{\markulf}[1]{{\color{orange}{MK: #1}}}
\newcommand{\mk}[1]{{\color{orange}{mk: #1}}}


% WIDEBAR

\makeatletter
\let\save@mathaccent\mathaccent
\newcommand*\if@single[3]{%
  \setbox0\hbox{${\mathaccent"0362{#1}}^H$}%
  \setbox2\hbox{${\mathaccent"0362{\kern0pt#1}}^H$}%
  \ifdim\ht0=\ht2 #3\else #2\fi
  }
%The bar will be moved to the right by a half of \macc@kerna, which is computed by amsmath:
\newcommand*\rel@kern[1]{\kern#1\dimexpr\macc@kerna}
%If there's a superscript following the bar, then no negative kern may follow the bar;
%an additional {} makes sure that the superscript is high enough in this case:
\newcommand*\widebar[1]{\@ifnextchar^{{\wide@bar{#1}{0}}}{\wide@bar{#1}{1}}}
%Use a separate algorithm for single symbols:
\newcommand*\wide@bar[2]{\if@single{#1}{\wide@bar@{#1}{#2}{1}}{\wide@bar@{#1}{#2}{2}}}
\newcommand*\wide@bar@[3]{%
  \begingroup
  \def\mathaccent##1##2{%
%Enable nesting of accents:
    \let\mathaccent\save@mathaccent
%If there's more than a single symbol, use the first character instead (see below):
    \if#32 \let\macc@nucleus\first@char \fi
%Determine the italic correction:
    \setbox\z@\hbox{$\macc@style{\macc@nucleus}_{}$}%
    \setbox\tw@\hbox{$\macc@style{\macc@nucleus}{}_{}$}%
    \dimen@\wd\tw@
    \advance\dimen@-\wd\z@
%Now \dimen@ is the italic correction of the symbol.
    \divide\dimen@ 3
    \@tempdima\wd\tw@
    \advance\@tempdima-\scriptspace
%Now \@tempdima is the width of the symbol.
    \divide\@tempdima 10
    \advance\dimen@-\@tempdima
%Now \dimen@ = (italic correction / 3) - (Breite / 10)
    \ifdim\dimen@>\z@ \dimen@0pt\fi
%The bar will be shortened in the case \dimen@<0 !
    \rel@kern{0.6}\kern-\dimen@
    \if#31
      \overline{\rel@kern{-0.6}\kern\dimen@\macc@nucleus\rel@kern{0.4}\kern\dimen@}%
      \advance\dimen@0.4\dimexpr\macc@kerna
%Place the combined final kern (-\dimen@) if it is >0 or if a superscript follows:
      \let\final@kern#2%
      \ifdim\dimen@<\z@ \let\final@kern1\fi
      \if\final@kern1 \kern-\dimen@\fi
    \else
      \overline{\rel@kern{-0.6}\kern\dimen@#1}%
    \fi
  }%
  \macc@depth\@ne
  \let\math@bgroup\@empty \let\math@egroup\macc@set@skewchar
  \mathsurround\z@ \frozen@everymath{\mathgroup\macc@group\relax}%
  \macc@set@skewchar\relax
  \let\mathaccentV\macc@nested@a
%The following initialises \macc@kerna and calls \mathaccent:
  \if#31
    \macc@nested@a\relax111{#1}%
  \else
%If the argument consists of more than one symbol, and if the first token is
%a letter, use that letter for the computations:
    \def\gobble@till@marker##1\endmarker{}%
    \futurelet\first@char\gobble@till@marker#1\endmarker
    \ifcat\noexpand\first@char A\else
      \def\first@char{}%
    \fi
    \macc@nested@a\relax111{\first@char}%
  \fi
  \endgroup
}
\makeatother


%%%%%%%%%%%%%%%%%%%%%%%%%%%%%%%%%%%%%
%     FONTS AND EDITING
%%%%%%%%%%%%%%%%%%%%%%%%%%%%%%%%%%%%%
%%%%%%%%%%%%%%%%%%%%%%%%%%%%%%%%%%%%%

\newcommand{\highlight}[1]{\colorbox{red!20}{$\displaystyle#1$}}
\definecolor{purple}{RGB}{139,0,139}
\newcommand{\ben}[1]{{\color{purple}{BD: #1}}}
\definecolor{darkgreen}{RGB}{0,128,0}
\newcommand{\chris}[1]{{\color{darkgreen}{CB: #1}}}
\definecolor{orange}{RGB}{210,105,30}
\newcommand{\geoffroy}[1]{{\color{orange}{GC: #1}}}
\newcommand{\todo}[1]{{\color{red}{TODO: #1}}}

\definecolor{darkblue}{RGB}{0,0,128}
\definecolor{midnightblue}{RGB}{0,0,210}
\definecolor{darkred}{RGB}{128,0,0}
\definecolor{figred}{RGB}{230,0,26}
\definecolor{figgreen}{RGB}{153,192,0}
\definecolor{figorange}{RGB}{245,163,0}
\definecolor{figblue}{RGB}{0,104,157}
\definecolor{figpurple}{RGB}{149,17,105}




%\theoremstyle{plain}
%\newtheorem{theorem}{Theorem}
%\newtheorem{proposition}{Proposition}
%\newtheorem{lemma}{Lemma}
%\newtheorem*{corollary}{Corollary}

%\newtheorem{definition}{Definition}
%\newtheorem{conjecture}{Conjecture}
%\newtheorem{algorithm}{Algorithm}
\newtheorem{construction}{Construction}
%\newtheorem{claim}{Claim}

%\theoremstyle{remark}
%\newtheorem*{remark}{Remark}
%\newtheorem*{note}{Note}
%\newtheorem{case}{Case}


%%%%%%%%%%%%%%%%%%%%%%%%%%%%%%%%%%%%%
%         STYLES
%%%%%%%%%%%%%%%%%%%%%%%%%%%%%%%%%%%%%

\newcommand{\R}{\mathbb{R}} \newcommand{\N}{\mathbb{N}}
\newcommand{\Z}{\mathbb{Z}} \newcommand{\Q}{\mathbb{Q}}
\newcommand{\card}[1]{\abs{#1}}
%\theoremstyle{definition} \newtheorem*{definition}{Definition}
%\newtheorem*{remark}{Remark} \newtheorem*{lemma}{Lemma}
%\newtheorem*{conjecture}{Conjecture} \newtheorem*{hint}{Hint}
%\newtheorem{claim}{Claim}
\newcommand{\cnt}{\textsf{ctr}}
\renewcommand{\H}{\textsf{H}}
\newcommand{\RO}{\textsf{RO}}
\newcommand{\mactag}{t}
\newcommand{\macinput}{x}
\newcommand{\RPO}{\textsf{RPO}}
\renewcommand{\O}[1]{\ensuremath{\mathsf{#1}}}
\newcommand{\E}{\textsf{E}}
\newcommand{\ininterface}[1]{{\mathsf{in}(#1)}} %{I^{in}_{#1}}
\newcommand{\outinterface}[1]{{\mathsf{out}(#1)}} %{I^{out}_{#1}}
\newcommand{\StyleModel}[1]{\V{{#1}}}

\newcommand{\kgen}{\StyleModel{kgen}}
\newcommand{\m}{\StyleModel{m}}
\newcommand{\se}{\StyleModel{se}}
\newcommand{\pke}{\StyleModel{pke}}
\renewcommand{\enc}{\StyleModel{enc}}
\renewcommand{\dec}{\StyleModel{dec}}
\renewcommand{\sig}{\StyleModel{sig}}
%\newcommand{\prp}{\StyleModel{prp}}
\newcommand{\nonce}{\StyleModel{nonce}}
\newcommand{\ac}{\StyleModel{ac}}
\newcommand{\ch}{\StyleModel{ch}}
\renewcommand{\mac}{\StyleModel{mac}}
\newcommand{\ver}{\StyleModel{ver}}
\renewcommand{\verify}{\StyleModel{ver}}
\newcommand{\receive}{\StyleModel{receive}}
\newcommand{\send}{\StyleModel{send}}
\newcommand{\unfcma}{\M{UNF-CMA}}
\newcommand{\unfcmaac}{\M{UNF-CMA-AC}}
\newcommand{\flunfcma}{\M{FL-UNF-CMA}}
\newcommand{\s}{\mathsf{s}}
%\newcommand{\pcassert}{\mathbf{assert}\;}
%\specialcomment{solution}{\textbf{Solution:}\;}{}
\newboolean{ifsolution}
\setboolean{ifsolution}{true}

\usepackage[breakable]{tcolorbox}

% if exercises only:
\newcommand{\sol}[1]{}

% if solution included:
%\newcommand{\sol}[1]{\begin{tcolorbox}[breakable]\solution{#1}\end{tcolorbox}}

\title{CS-E4340 Cryptography: Exercise Sheet 6\\%[0.5\baselineskip]
\large ---Public-key Encryption %Key Exchange 
\& Signature Schemes---}
\author{}\institute{}%Submission deadline: September 12, 2022, 11:30, via MyCourses}\institute{}
%\author{Christopher Brzuska, Pihla Karanko}
%\institute{
%			 Aalto University, Finland
%}
%\date{\today}
\date{}

\begin{document}
\maketitle

\medskip
\textbf{Submission deadline: October 31, 2022, 11:30, via MyCourses}

\medskip
\noindent
Each exercise can give up to two participation points, 2 for a mostly correct solution and 1 point for a good attempt. Overall, the exercise sheet gives at most 4 participation points. We encourage to \textbf{choose} exercises which seem \textbf{interesting} and/or adequately challenging to you. %In particular, you can skip some exercises, if you like. Your teaching assistant will carefully study your ideas and provide helpful suggestions. 


\medskip
\noindent
Exercise Sheet 6 is intended to help...
\begin{itemize}
\item[(a)] ...understand the \emph{definition} of IND-CPA security of public-key encryption (Ex. 1 $\&$ Ex.2).
\item[(b)] ...understand the \emph{definition} of UNF-CMA security for digital signature schemes (Ex. 3 $\&$ 4).
\item[(c)] ...familiarize yourself with textbook RSA and its limitations (Ex. 2 $\&$ Ex.3).
\item[(d)] ...reflect on the relation between signature schemes and public-key encryption (Ex. 3).
\item[(e)] ...reflect on the relation between signature schemes and one-way functions (Ex. 4).
%\item[(e)] ...develop intuition for security and attacks on key exchange protocols (Ex. 3).
\end{itemize}

%\begin{description}
%\item[Exercise 1] helps to understand IND-CPA security.
%\item[Exercise 2] helps understand the notion of one-wayness and is the \underline{\textbf{most important}} exercise on this sheet. We %warmly encourage it.
%\item[Exercise 3] gives an opportunity to practice writing \emph{inverters}\footnote{For OWFs, the term \emph{adversary} and \emph{inverter} are synonymous, because an adversary against a OWF tries to invert.} for some easy-to-invert functions, i.e. bad one-way function candidates.
%\item[Ex. 4 \& Ex. 5] are advanced exercises, where you are asked to provide an attack on a generic one-way function counterexample and analyze the probability of the attack.
%\end{description}


%\begin{abstract}
%The goal of this exercise sheet is to familiarize yourself with the security of public-key encryption schemes
%(Exercise 1 and Exercise 2). Then, there is an exercise about \emph{hybrid} public-key encryption which is
%the way in which public-key encryption is used in practice: Namely, we first use a public-key encryption scheme
%to encrypt a symmetric-key and then encrypt the message using the symmetric-key and the symmetric-key encryption
%scheme. This is more efficient, since public-key encryption is slow and symmetric-key encryption is fast. So, if
%the message is long, we gain efficiency. The proof might be a little involved, so we marked this exercise as
% advanced (Exercise 4). If you are curious to look a bit more into RSA, you can have a look at Exercise 5. And if you 
%are curious about oracle separations, then have a look at Exercise 3. If oracle separations or RSA are not your
%cup of tea, the rest of the course will not use these. It's more for context and breadth that we include this.
%\end{abstract}

\begin{graded}[Deterministic PKE is insecure]
On Ex. Sheet 5, we showed that symmetric-key encryption is not IND-CPA-secure and described
an attack using \emph{two} $\O{ENC}$ queries. Let $\pke_{\text{weak}}$ be a candidate
PKE which satisfies the syntax of PKE and where $\pke_{\text{weak}}.\enc$ is deterministic.\\

\noindent
\textbf{Task:} Describe a PPT adversary $\adv$ in pseudocode which breaks the IND-CPA security of
$\pke_{\text{weak}}$ and only makes a $\O{GETPK}$ and a
 \emph{a single} $\O{ENC}$ query. Analyze the
success probability of your adversary and show that it is non-negligible.\\

%\noindent
%\textbf{Hint:} Can you think of an adversary which has advantage is $1$?
\end{graded}

		\begin{figure}
		\centering
			\begin{pchstack}
			\begin{pcvstack}
				\procedure{$\pke_\text{RSA}.\kgen()$}
				{
				\text{sample two big random primes }p,q\\
				N \gets pq\\
				\lambda \gets \text{lcm}(p-1,q-1)\\
				\text{choose $e > 1$ that is coprime with }\lambda\\
				d \gets e^{-1} \mod \lambda\\
				sk \gets (d,N);\,
				pk \gets (e,N)\\
				\pcreturn pk
				}
				\pcvspace
					\procedure{$s_\text{RSA}.\kgen()$}
				{
				[\text{same as }\pke_\text{RSA}.\kgen]
				}
			\end{pcvstack}
\pchspace
			\begin{pcvstack}
				\procedure{$\pke_\text{RSA}.\enc(pk,m)$}
				{
					(e,N) \gets pk\\[0.16\baselineskip]
					c \gets m^e \mod N\\
					\pcreturn c\\\\
				}
				\pcvspace
				\procedure{$s_\text{RSA}.\sig(sk,m)$}
				{
					(d,N) \gets sk\\
					\sigma \gets m^d \mod N\\
					\pcreturn \sigma
				}
			\end{pcvstack}
\pchspace
			\begin{pcvstack}
				\procedure{$\pke_\text{RSA}.\dec(sk,c)$}
				{(d,N) \gets sk\\
				m \gets c^{d}\mod N\\
					\pcreturn m\\\\
				}
				\pcvspace
				\procedure{$s_\text{RSA}.\ver(pk,m,\sigma)$}
				{(e,N) \gets pk\\
				m' \gets \sigma^e\mod N\\
					\pcreturn m = m'
				}
			\end{pcvstack}
			\end{pchstack}
\caption{Textbook RSA (insecure)\label{fig:textbookrsa}}
\end{figure}


\begin{graded}[RSA: Public-Key encryption]
Fig.~\ref{fig:textbookrsa} describes the RSA encryption scheme $\pke_\text{RSA}$ and RSA signature scheme $s_\text{RSA}$ as some textbooks, discrete mathematics courses and the RSA Wikipedia article\footnote{\url{https://en.wikipedia.org/wiki/RSA_(cryptosystem)}} (see Section 3) do. $\pke_\text{RSA}$ and $s_\text{RSA}$ are thus called \emph{textbook RSA}. This exercise explores
why textbook RSA should not be used \emph{as is} in practice and other confusions/misconceptions emerging from textbook RSA
in popular literature.

\medskip
\noindent
\textbf{Task 1:}
Compute $\pke_\text{RSA}.\dec(\sk,c)$ for ciphertext $c=61$ and secret-key $\sk=(119,37)$.
(The public-key here is $\pk=(119,13)$, but it is used for encryption only, not decryption.)

\medskip
\noindent
\textbf{Task 2:}
Prove that $\pke_\text{RSA}$ is not IND-CPA-secure by giving a PPT adversary $\adv$ against
the IND-CPA security of $\pke_\text{RSA}$ in pseudo-code. (You can omit the probability analysis.)
\end{graded}



\begin{graded}[RSA: Signing vs. Public-Key encryption]
As in the previous exercise, see Fig.~\ref{fig:textbookrsa} for the textbook RSA encryption scheme $\pke_\text{RSA}$ and textbook RSA signature scheme $s_\text{RSA}$.

\textbf{Task 1:}
Prove that $\sig_\text{RSA}$ is not UNF-CMA-secure by giving a PPT adversary $\adv$ against
the UNF-CMA security of $\sig_\text{RSA}$ in pseudo-code. (You can omit the probability analysis.)

\medskip
\noindent
\textbf{Task 2:} Some sources describe signature schemes as the ``opposite'' or ``inverse'' of
		encryption. The underlying idea is that textbook RSA encryption $\pke_\text{RSA}.\enc$ encrypts using the public-key, and the
		textbook RSA signature schemes $s_\text{RSA}.\sig$
		%(the insecure variant, which just signs the plain message without pre-processing)
		%which should
		%absolutely hash the message before feeding it into the RSA function, because else, they are
		%insecure, too) 
		signs using the secret-key.
		
		Reflect whether this intuition generalizes. Can every signature scheme
		be transformed into a public-key encryption scheme? Justify your belief.
	\end{graded}
		

\begin{graded}[OWFs $\Rightarrow$ SIG]
Show that if $f$ is an injective one-way function, then Lamport's signature scheme $s_f$
is one-time UNF-CMA-secure (1-UNF-CMA) for messages of length $\lambda$. See Fig.~\ref{fig:lamport}
for $s_f$ and see below for the definition of 1-UNF-CMA.
\end{graded}

		\begin{figure}
		\centering
			\begin{pchstack}
			\begin{pcvstack}
					\procedure{$s_\text{f}.\kgen(1^\lambda)$}
				{
				\pcfor i=1..\lambda\\
				\pcind x_0^i\sample\bin^\lambda\\
				\pcind x_1^i\sample\bin^\lambda\\
				\pcind y_0^i\gets f(x_0^i)\\
				\pcind y_0^i\gets f(x_1^i)\\
				\sk\gets\begin{minipage}{0.05\linewidth}$\left(\begin{minipage}{10\linewidth}
					$x_0^1,..,x_0^\lambda$\\
					$x_1^1,..,x_1^\lambda$\end{minipage}\right)$\end{minipage}\\
				\pk\gets \begin{minipage}{0.05\linewidth}$\left(\begin{minipage}{10\linewidth} $y_0^1,..,y_0^\lambda$\\$y_1^1,..,y_1^\lambda$\end{minipage}\right)$\end{minipage}
				}\\
				\pcreturn (\sk,\pk)
			\end{pcvstack}
\pchspace
			\begin{pcvstack}
				\procedure{$s_f.\sig(\sk,m)$}
				{
					\mathsf{parse}\,\sk\text{ as}\\
				\begin{minipage}{0.041\linewidth}$\left(\begin{minipage}{10\linewidth}
					$x_0^1,..,x_0^\lambda$\\
					$x_1^1,..,x_1^\lambda$\end{minipage}\right)$\end{minipage}\\
				\pcfor i=1..\lambda\\
				\pcind z^i\gets x^i_{m[i]}\\
				\pccomment{m[i]\text{ is }i\text{th bit of }m}\\
				\sigma\gets (z^1,..,z^\lambda)\\
				\pcreturn \sigma
				}
			\end{pcvstack}
\pchspace
			\begin{pcvstack}
				\procedure{$s_f.\ver(\pk,m,\sigma)$}
				{					\mathsf{parse}\,\pk\text{ as}\\
				\begin{minipage}{0.041\linewidth}$\left(\begin{minipage}{10\linewidth}
					$y_0^1,..,y_0^\lambda$\\
					$y_1^1,..,y_1^\lambda$\end{minipage}\right)$\end{minipage}\\
					(z^1,..,z^\lambda)\gets \sigma\\
			\pcfor i=1..\lambda\\
				\pcind \pcif f(z^i)\neq x^i_{m[i]}:\\
				\pcind\pcind\pcreturn 0\\
				\pcreturn 1
				}
			\end{pcvstack}
			\end{pchstack}
\caption{Lamport's one-time signature scheme for messages of length $\lambda$.\label{fig:lamport}}
\end{figure}

\clearpage

\hspace{-4cm}
\scalebox{0.8}{
\begin{minipage}{1.8\textwidth}
	\begin{minipage}{0.32\textwidth}
    \begin{align*} 
      \textbf{Public-key encryption scheme } \pke:\,(\pk,\sk)\sample&\pke.\kgen(1^\lambda) \\
      c\sample&\pke.\enc(\pk,x) \\
      x\sample &\pke.\dec(\sk,c)
    \end{align*}
    \textbf{Correctness:} $\forall (\pk,\sk)\sample\pke.\kgen(1^\lambda)$, $\forall x\in\bin^*:$
		$\forall c\sample\pke.\enc(\pk,x)$,  $\pke.\dec(\sk,c)=x$.

\medskip
\noindent
\textbf{IND-CPA Security}: $\forall$ PPT $\adv$
\begin{align*}
|&\prob{1=\adv\rightarrow \M{Gind\text{-}cpa}^0_\pke}\\
-&\prob{1=\adv\rightarrow \M{Gind\text{-}cpa}^1_\pke}| \text{ is negligible in }\lambda.
\end{align*}

\medskip
\medskip\medskip\medskip\medskip\medskip\medskip\medskip\medskip\medskip

\begin{minipage}{0.4\textwidth}
  \begin{pchstack}
        \begin{pcvstack}
          \underline{\underline{$\M{Gind\text{-}cpa}_\pke^0$}}\\
                          \\
    \procedure{$\O{GETPK}()$}{
		    \pcif \pk=\bot:\\
				\pcind  (\pk,\sk)\sample\pke.\kgen(1^\lambda)\\
      \pcreturn \pk}
										\pcvspace
		  \procedure{$\O{ENC}(x)$}{
			  \pcif \pk=\bot:\\
				\pcind (\pk,\sk)\sample \pke.\kgen(1^\lambda)\\
				\\
			 c\sample \pke.\enc(\pk,x)\\
			\pcreturn c}
		\end{pcvstack}
              \pchspace
  \begin{pcvstack}
          \underline{\underline{$\M{Gind\text{-}cpa}_\pke^1$}}\\
                                \\
    \procedure{$\O{GETPK}()$}{
		    \pcif \pk=\bot:\\
				\pcind  (\pk,\sk)\sample\pke.\kgen(1^\lambda)\\
      \pcreturn \pk}
										\pcvspace
		  \procedure{$\O{ENC}(x)$}{
			  \pcif \pk=\bot:\\
				\pcind (\pk,\sk)\sample \pke.\kgen(1^\lambda)\\
				x'\gets 0^{\abs{x}}\\
			 c\sample \pke.\enc(\pk,x')\\
			\pcreturn c\\\\\\\\\\\\\\\\}
  \end{pcvstack}
  \end{pchstack}
  \end{minipage}
	\end{minipage}
\hspace{0.5cm}
	\begin{minipage}{0.32\textwidth}
    \begin{align*} 
      \textbf{Signature scheme }\s:\;\;\;\,(\pk,\sk)\sample&\s.\kgen(1^\lambda) \\
      \sigma\sample&\s.\sig(\sk,x) \\
      0/1\sample &\s.\ver(\pk,x,\sigma)
    \end{align*}
    \textbf{Correctness:} $\forall (\pk,\sk)\sample\s.\kgen(1^\lambda)$, $\forall x\in\bin^*:$
		$\forall\sigma\sample\s.\sig(\sk,x)$,  $\s.\ver(\pk,x,\sigma)=1$.

\medskip
\noindent
\textbf{UNF-CMA Security}: $\forall$ PPT $\adv$
\begin{align*}
|&\prob{1=\adv\rightarrow \M{Gunf\text{-}cma}^0_\s}\\
-&\prob{1=\adv\rightarrow \M{Gunf\text{-}cma}^1_\s}| \text{ is negligible in }\lambda.
\end{align*}

\medskip
\noindent
\textbf{1-UNF-CMA Security for Lamport}: $\forall$ PPT $\adv$
\begin{align*}
|&\prob{1=\adv\rightarrow \M{1-Gunf\text{-}cma}^0_\s}\\
-&\prob{1=\adv\rightarrow \M{1-Gunf\text{-}cma}^1_\s}| \text{ is negligible in }\lambda.
\end{align*}



\begin{minipage}{0.4\textwidth}
  \begin{pchstack}
        \begin{pcvstack}
          \underline{\underline{$\M{Gunf\text{-}cma}_\s^0$}}\\
                          \\
    \procedure{$\O{GETPK}()$}{
		    \pcif \pk=\bot:\\
				\pcind  (\pk,\sk)\sample\s.\kgen(1^\lambda)\\
      \pcreturn \pk}
										\pcvspace
    \procedure{$\O{SIG}(x)$}{
		  \\%\gamechange{$\pcassert \sigma=\bot$}\\ 
		  \\%\gamechange{$\pcassert \abs{x}=\lambda$}\\ 
		  \pcif \pk=\bot:\\
			\pcind (\pk,\sk)\sample\s.\kgen(1^\lambda)\\
      \sigma\sample \s.\sig(\sk,x)\\
      \\
      \pcreturn \sigma}
   \pcvspace
		\procedure{$\O{VERIFY}(x,\sigma)$}{
		    \pcif \pk=\bot:\\
				\pcind  (\pk,\sk)\sample\s.\kgen(1^\lambda)\\
      d\gets \s.\ver(\pk,x,\sigma)\\
      \pcreturn d}
  \end{pcvstack}
              \pchspace
  \begin{pcvstack}
          \underline{\underline{$\M{Gunf\text{-}cma}_\s^1$}}\\
                                \\
    \procedure{$\O{GETPK}()$}{
		    \pcif \pk=\bot:\\
				\pcind  (\pk,\sk)\sample\s.\kgen(1^\lambda)\\
      \pcreturn \pk}
          \pcvspace
    \procedure{$\O{SIG}(x)$}{
		 \\% \gamechange{$\pcassert \sigma=\bot$}\\ 
		  \\% \gamechange{$\pcassert \abs{x}=\lambda$}\\ 
		    \pcif \pk=\bot:\\
				\pcind  (\pk,\sk)\sample\s.\kgen(1^\lambda)\\
      \sigma\sample \s.\sig(\sk,x)\\
      \mathcal{L}\gets\mathcal{L}\cup\{(x,\sigma)\}\\
      \pcreturn \sigma}
          \pcvspace
    \procedure{$\O{VERIFY}(x,\sigma)$}{
          \pcif (x,\sigma)\in\mathcal{L}:\\
          \pcind \pcreturn 1\\
          \\
          \pcreturn 0}
  \end{pcvstack}
\pchspace
        \begin{pcvstack}
          \underline{\underline{$\M{1-Gunf\text{-}cma}_\s^0$}}\\
                          \\
    \procedure{$\O{GETPK}()$}{
		    \pcif \pk=\bot:\\
				\pcind  (\pk,\sk)\sample\s.\kgen(1^\lambda)\\
      \pcreturn \pk}
										\pcvspace
    \procedure{$\O{SIG}(x)$}{
		  \gamechange{$\pcassert \sigma=\bot$}\\ 
		  \gamechange{$\pcassert \abs{x}=\lambda$}\\ 
		  \pcif \pk=\bot:\\
			\pcind (\pk,\sk)\sample\s.\kgen(1^\lambda)\\
      \sigma\sample \s.\sig(\sk,x)\\
      \\
      \pcreturn \sigma}
   \pcvspace
		\procedure{$\O{VERIFY}(x,\sigma)$}{
		    \pcif \pk=\bot:\\
				\pcind  (\pk,\sk)\sample\s.\kgen(1^\lambda)\\
      d\gets \s.\ver(\pk,x,\sigma)\\
      \pcreturn d}
  \end{pcvstack}
              \pchspace
  \begin{pcvstack}
          \underline{\underline{$\M{1-Gunf\text{-}cma}_\s^1$}}\\
                                \\
    \procedure{$\O{GETPK}()$}{
		    \pcif \pk=\bot:\\
				\pcind  (\pk,\sk)\sample\s.\kgen(1^\lambda)\\
      \pcreturn \pk}
          \pcvspace
    \procedure{$\O{SIG}(x)$}{
		  \gamechange{$\pcassert \sigma=\bot$}\\ 
		  \gamechange{$\pcassert \abs{x}=\lambda$}\\ 
		    \pcif \pk=\bot:\\
				\pcind  (\pk,\sk)\sample\s.\kgen(1^\lambda)\\
      \sigma\sample \s.\sig(\sk,x)\\
      \mathcal{L}\gets\mathcal{L}\cup\{(x,\sigma)\}\\
      \pcreturn \sigma}
          \pcvspace
    \procedure{$\O{VERIFY}(x,\sigma)$}{
          \pcif (x,\sigma)\in\mathcal{L}:\\
          \pcind \pcreturn 1\\
          \\
          \pcreturn 0}
  \end{pcvstack}
  \end{pchstack}
  \end{minipage}
	\end{minipage}
	\end{minipage}
}

\end{document}
