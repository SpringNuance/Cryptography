\newif\iffull\fullfalse% do not change flags here
\newif\ifdraft\draftfalse% do not change flags here

\documentclass[envcountsame,runningheads,notitlepage]{../llncs}
\pagestyle{plain}

\usepackage[letterpaper,hmargin=3.5cm,vmargin=3cm]{geometry}
\usepackage[T1]{fontenc}
\usepackage{lmodern}
\usepackage{xifthen}
\usepackage{amsmath}
\usepackage{amsfonts}
\usepackage{amssymb}
\usepackage{lscape}
%\usepackage{amsthm}
\usepackage{xspace}
\usepackage{enumerate}
\usepackage{tikz}
\usepackage{hyperref}
\usepackage{color,soul}
\usepackage{tikz}
\usetikzlibrary{shapes,arrows, calc}
\usepackage{graphicx}
\usepackage{array}
\usepackage{subcaption}
\usepackage[probability,adversary,sets,notions,operators,complexity,keys,primitives,asymptotics,advantage]{cryptocode}
\usepackage{savesym}
\usepackage{relsize}
\savesymbol{cb}
\usepackage{combelow}
\usepackage[normalem]{ulem}
\usepackage{wrapfig}
\newtheorem{crossed}{Esitehtävä}
\usepackage{enumitem}

\let\proof\relax
\let\endproof\relax
\usepackage{amsthm}

\theoremstyle{definition}
\newtheorem{graded}[crossed]{Exercise}
\newtheorem{challenge}[crossed]{Challenge exercise}

% Colors
\newcommand\maybecolor[1]{\color{#1}}
\definecolor{lightblue}{rgb}{0.2,0.2,1.0}
\definecolor{dkred}{rgb}{0.5, 0.0,0.0}
\definecolor{lightgrey}{rgb}{0.8,0.8,0.8}
\definecolor{dkblue}{rgb}{0,0.1,0.5}
\definecolor{dkgreen}{rgb}{0,0.4,0}
\definecolor{dkred}{rgb}{0.6,0,0}
\definecolor{linkColor}{rgb}{0,0,0.5}
\definecolor{lightgray}{rgb}{.9,.9,.9}
\definecolor{darkgray}{rgb}{.4,.4,.4}
\definecolor{purple}{rgb}{0.65, 0.12, 0.82}

\ifdraft
\newcommand{\fullcolor}{\color{blue}}
\else
\newcommand{\fullcolor}{\color{black}}
\fi
\newcommand{\blk}{\color{black}}
\newcommand{\dkg}{\color{dkgreen}}

% Variable and procedures names
\renewcommand{\O}[1]{\ensuremath{\mathsf{#1}}}
\newcommand{\V}[1]{\ensuremath{\mathit{#1}}}
\newcommand{\M}[1]{\ensuremath{\text{\tt#1}}}
\def \G {\mathcal{G}}
\newcommand{\hLen}{n}




%\newcommand{\markulf}[1]{{\color{orange}{MK: #1}}}
\newcommand{\mk}[1]{{\color{orange}{mk: #1}}}


% WIDEBAR

\makeatletter
\let\save@mathaccent\mathaccent
\newcommand*\if@single[3]{%
  \setbox0\hbox{${\mathaccent"0362{#1}}^H$}%
  \setbox2\hbox{${\mathaccent"0362{\kern0pt#1}}^H$}%
  \ifdim\ht0=\ht2 #3\else #2\fi
  }
%The bar will be moved to the right by a half of \macc@kerna, which is computed by amsmath:
\newcommand*\rel@kern[1]{\kern#1\dimexpr\macc@kerna}
%If there's a superscript following the bar, then no negative kern may follow the bar;
%an additional {} makes sure that the superscript is high enough in this case:
\newcommand*\widebar[1]{\@ifnextchar^{{\wide@bar{#1}{0}}}{\wide@bar{#1}{1}}}
%Use a separate algorithm for single symbols:
\newcommand*\wide@bar[2]{\if@single{#1}{\wide@bar@{#1}{#2}{1}}{\wide@bar@{#1}{#2}{2}}}
\newcommand*\wide@bar@[3]{%
  \begingroup
  \def\mathaccent##1##2{%
%Enable nesting of accents:
    \let\mathaccent\save@mathaccent
%If there's more than a single symbol, use the first character instead (see below):
    \if#32 \let\macc@nucleus\first@char \fi
%Determine the italic correction:
    \setbox\z@\hbox{$\macc@style{\macc@nucleus}_{}$}%
    \setbox\tw@\hbox{$\macc@style{\macc@nucleus}{}_{}$}%
    \dimen@\wd\tw@
    \advance\dimen@-\wd\z@
%Now \dimen@ is the italic correction of the symbol.
    \divide\dimen@ 3
    \@tempdima\wd\tw@
    \advance\@tempdima-\scriptspace
%Now \@tempdima is the width of the symbol.
    \divide\@tempdima 10
    \advance\dimen@-\@tempdima
%Now \dimen@ = (italic correction / 3) - (Breite / 10)
    \ifdim\dimen@>\z@ \dimen@0pt\fi
%The bar will be shortened in the case \dimen@<0 !
    \rel@kern{0.6}\kern-\dimen@
    \if#31
      \overline{\rel@kern{-0.6}\kern\dimen@\macc@nucleus\rel@kern{0.4}\kern\dimen@}%
      \advance\dimen@0.4\dimexpr\macc@kerna
%Place the combined final kern (-\dimen@) if it is >0 or if a superscript follows:
      \let\final@kern#2%
      \ifdim\dimen@<\z@ \let\final@kern1\fi
      \if\final@kern1 \kern-\dimen@\fi
    \else
      \overline{\rel@kern{-0.6}\kern\dimen@#1}%
    \fi
  }%
  \macc@depth\@ne
  \let\math@bgroup\@empty \let\math@egroup\macc@set@skewchar
  \mathsurround\z@ \frozen@everymath{\mathgroup\macc@group\relax}%
  \macc@set@skewchar\relax
  \let\mathaccentV\macc@nested@a
%The following initialises \macc@kerna and calls \mathaccent:
  \if#31
    \macc@nested@a\relax111{#1}%
  \else
%If the argument consists of more than one symbol, and if the first token is
%a letter, use that letter for the computations:
    \def\gobble@till@marker##1\endmarker{}%
    \futurelet\first@char\gobble@till@marker#1\endmarker
    \ifcat\noexpand\first@char A\else
      \def\first@char{}%
    \fi
    \macc@nested@a\relax111{\first@char}%
  \fi
  \endgroup
}
\makeatother


%%%%%%%%%%%%%%%%%%%%%%%%%%%%%%%%%%%%%
%     FONTS AND EDITING
%%%%%%%%%%%%%%%%%%%%%%%%%%%%%%%%%%%%%
%%%%%%%%%%%%%%%%%%%%%%%%%%%%%%%%%%%%%

\newcommand{\highlight}[1]{\colorbox{red!20}{$\displaystyle#1$}}
\definecolor{purple}{RGB}{139,0,139}
\newcommand{\ben}[1]{{\color{purple}{BD: #1}}}
\definecolor{darkgreen}{RGB}{0,128,0}
\newcommand{\chris}[1]{{\color{darkgreen}{CB: #1}}}
\definecolor{orange}{RGB}{210,105,30}
\newcommand{\geoffroy}[1]{{\color{orange}{GC: #1}}}
\newcommand{\todo}[1]{{\color{red}{TODO: #1}}}

\definecolor{darkblue}{RGB}{0,0,128}
\definecolor{midnightblue}{RGB}{0,0,210}
\definecolor{darkred}{RGB}{128,0,0}
\definecolor{figred}{RGB}{230,0,26}
\definecolor{figgreen}{RGB}{153,192,0}
\definecolor{figorange}{RGB}{245,163,0}
\definecolor{figblue}{RGB}{0,104,157}
\definecolor{figpurple}{RGB}{149,17,105}




%\theoremstyle{plain}
%\newtheorem{theorem}{Theorem}
%\newtheorem{proposition}{Proposition}
%\newtheorem{lemma}{Lemma}
%\newtheorem*{corollary}{Corollary}

%\newtheorem{definition}{Definition}
%\newtheorem{conjecture}{Conjecture}
%\newtheorem{algorithm}{Algorithm}
\newtheorem{construction}{Construction}
%\newtheorem{claim}{Claim}

%\theoremstyle{remark}
%\newtheorem*{remark}{Remark}
%\newtheorem*{note}{Note}
%\newtheorem{case}{Case}


%%%%%%%%%%%%%%%%%%%%%%%%%%%%%%%%%%%%%
%         STYLES
%%%%%%%%%%%%%%%%%%%%%%%%%%%%%%%%%%%%%

\newcommand{\R}{\mathbb{R}} \newcommand{\N}{\mathbb{N}}
\newcommand{\Z}{\mathbb{Z}} \newcommand{\Q}{\mathbb{Q}}
\newcommand{\card}[1]{\abs{#1}}
%\theoremstyle{definition} \newtheorem*{definition}{Definition}
%\newtheorem*{remark}{Remark} \newtheorem*{lemma}{Lemma}
%\newtheorem*{conjecture}{Conjecture} \newtheorem*{hint}{Hint}
%\newtheorem{claim}{Claim}
\newcommand{\cnt}{\textsf{ctr}}
\renewcommand{\H}{\textsf{H}}
\newcommand{\RO}{\textsf{RO}}
\newcommand{\RPO}{\textsf{RPO}}
\renewcommand{\O}[1]{\ensuremath{\mathsf{#1}}}
\newcommand{\E}{\textsf{E}}
\newcommand{\ininterface}[1]{{\mathsf{in}(#1)}} %{I^{in}_{#1}}
\newcommand{\outinterface}[1]{{\mathsf{out}(#1)}} %{I^{out}_{#1}}
\newcommand{\StyleModel}[1]{\V{{#1}}}
\newtheorem*{hint}{Hint}

\newcommand{\kgen}{\StyleModel{kgen}}
\newcommand{\m}{\StyleModel{m}}
\newcommand{\se}{\StyleModel{se}}
\renewcommand{\enc}{\StyleModel{enc}}
\renewcommand{\dec}{\StyleModel{dec}}
%\newcommand{\prp}{\StyleModel{prp}}
\newcommand{\nonce}{\StyleModel{nonce}}
\newcommand{\ac}{\StyleModel{ac}}
\newcommand{\ch}{\StyleModel{ch}}
\renewcommand{\mac}{\StyleModel{mac}}
\newcommand{\ver}{\StyleModel{ver}}
\renewcommand{\verify}{\StyleModel{ver}}
\newcommand{\receive}{\StyleModel{receive}}
\newcommand{\send}{\StyleModel{send}}
\newcommand{\unfcma}{\M{UNF-CMA}}
\newcommand{\unfcmaac}{\M{UNF-CMA-AC}}
\newcommand{\flunfcma}{\M{FL-UNF-CMA}}
%\newcommand{\pcassert}{\mathbf{assert}\;} Please update cryptocode package if you get an error saying that pcassert is not defined.
%\specialcomment{solution}{\textbf{Solution:}\;}{}
\newboolean{ifsolution}
\setboolean{ifsolution}{true}

\usepackage[breakable]{tcolorbox}

% if exercises only:
\newcommand{\sol}[1]{}

% if solution included:
%\newcommand{\sol}[1]{\begin{tcolorbox}[breakable]\solution{#1}\end{tcolorbox}}

\title{CS-E4340 Cryptography: Exercise Sheet 3}
\author{}\institute{}
%\author{Osama Abuzaid, Christopher Brzuska, Pihla Karanko, Miikka Tiainen}
%\institute{
%			 Aalto University, Finland
%}
%\date{\today}

\begin{document}
%\maketitle


\maketitle

\noindent
\textbf{Submission Deadline: September 26, 11:30 via MyCourses}

\medskip
\noindent
Each exercise can give up to two participation points, 2 for a mostly correct solution and 1 point for a good attempt. Overall, the exercise sheet gives at most 4 participation points.

\medskip
\noindent
Exercise Sheet 3 is intended to help...
\begin{itemize}
\item[(a)] ...understand the definition of pseudorandom functions (PRFs).
\item[(b)] ...understand the difference between a PRF and a pseudorandom generator (PRG), which we considered before.
\item[(c)] ...familiarize yourself with the notion of a reduction (continued).
\item[(d)] ...familiarize yourself with the notion of a negligible function.
\end{itemize}





\begin{description}
\item[Exercise 1] shows PRFs \textit{do} hide their input unlike some other primitives we saw thus far.
\item[Ex. 2 \& Ex. 3] concern properties of PRFs and PRGs.
\item[Exercise 2] shows that the existence of PRFs implies the existence of PRGs.
\item[Exercise 3] shows how a quadratic key PRF and PRG can be used to obtain a standard PRF.
\item[Exercise 4] is intended to help with the notions of negligible functions and why they are a convenient notion of a ``small'' function.
%\item[Exercise 5] shows that the existence of PRGs implies the existence of PRFs. (It is possible to construct a PRF using PRGs!)
% omitted since there is no model solution.
\end{description}

\clearpage


%TODO negligibility exercise + additional exercises (GGM or packages? )
%TODO fix/check abstract
%TODO fix/check notation in proofs and exercises



\begin{graded}[\textbf{PRFs hide their input}]
Let $f$ be a $(\lambda,\lambda)$-PRF, and consider the following transformations:
\begin{description}
\item[(a)] $h_1(k,x):=f(k,0||x_{2..\abs{x}})$.
\item[(b)] $h_2(k,x):=x_1||f(k,x)_{2..\abs{x}}$.
\end{description}
\textbf{Task:} Show that $h_1$ and $h_2$ are not PRFs (even though $f$ is a PRF).\\
\textbf{Hint:} Find an efficient adversary $\adv$ such that $\mathbf{Adv}^{\mathsf{PRF}}_{h,\adv}(1^\lambda)$ is non-negligible. (In fact, we can even give an adversary which has advantage almost $1$, but this is not required to solve this exercise.)
\end{graded}

\begin{graded}[\textbf{PRFs imply PRGs}]
  Let $f$ be a PRF. Define
	\[G(z):=f(z,0^{\abs{z}})||f(z,1^{\abs{z}}).\]

	\noindent
	\textbf{Task:}
	Prove via reduction, that $G$ is a
  PRG with output length $\abs{G(z)}=2\abs{z}$.
\end{graded}

\begin{graded}\textbf{(Key Expansion)}
A \emph{quadratic-key} $(\lambda,\lambda)$-PRF $f_q$ is a PRF which maps a key $k$ of length $\lambda^2$,
and an input $x$ of length $\lambda$ to an output $y$ of length $\lambda$. Let $f_q$ be a quadratic-key PRF.
Let $G$ be a PRG with $\abs{G(z)}=\abs{z}^2$. Define
\[f(k,x):=f_q(G(k),x)\]
\textbf{Task:} Show that $f$ is a (standard) PRF.\\
\textbf{Hint:} Given a successful distinguisher for the PRF $f$, show that one of the following is true: (i) There exists a successful distinguisher for the PRG $G$ or (ii) there exists a successful distinguisher for the quadradic-key PRF $f_q$.
\end{graded}

%\newpage
\begin{graded}[\textbf{Negligible Functions}]
Recall the definition of negligible functions.
\\
\\
\textbf{Definition 1} A function \(\nu: \NN \to \RR^+_0\) is \textit{negligible} if for all constants \(c\) there exists a natural number \(N \in \NN\) such that for all \(n > N\) it holds that \(\nu(n) < \frac{1}{n^c}.\)\\
\\
Closer to the verbal description of negligible function given in the lecture notes for lecture 3, we may like to define negligible functions as follows:\\
\\
\textbf{Definition 1'} A function \(\nu: \NN \to \RR^+_0\) is \textit{negligible} if for all positive polynomials \(p\) there exists a natural number \(N \in \NN\) such that for all \(n > N\) it holds that \(\nu(n) < \frac{1}{p(n)}.\)\\
\\
Prove at least two out of (a), (b) and (c):
\begin{description}
\item[(a)] Definitions 1 and 1' of negligible functions are equivalent.
\item[(b)] The following are true:
	\begin{itemize}
	\item[(i)] The sum of two negligible functions is negligible.
	\item[(ii)] Multiplying a negligible function by an (arbitrary) positive polynomial yields a negligible function.
\end{itemize}
\textbf{Hint:} You may use either of the two definitions in your proof.
\item[(c)] (Challenging) There exists a sequence of negligible functions $\nu_\lambda:\NN\rightarrow[0,1]$ such that the function $\mu(\lambda):=\sum_{i=1}^\lambda \nu_i(\lambda)$ is the constant $1$ function, i.e., for all $\lambda\in\NN$, it holds that $\mu(\lambda)=1$.

\textbf{Hint:} Use diagonalization.
\end{description}
\end{graded}


\end{document}

%%% Local Variables:
%%% mode: latex
%%% TeX-master: t
%%% End:
