\documentclass[a4paper,table,dvipsnames]{article}
\usepackage{lmodern}
\usepackage[utf8]{inputenc}
\usepackage[T1]{fontenc}
\usepackage[english]{babel}
\usepackage{latexsym}
\usepackage{amsmath,amsfonts,amssymb}
\usepackage{mathpartir}
\usepackage{amsthm}
\usepackage{tikz}
\usepackage{enumitem}
\usepackage[probability,adversary,sets,operators,primitives]{cryptocode}
\usepackage{fancyhdr}
\usepackage{hyperref}
\usepackage{wrapfig}
%\usepackage[table]{xcolor}
%\usepackage[margin=2.5in]{geometry}

\pagestyle{fancy}
\usetikzlibrary{shapes,arrows,positioning}
\tikzstyle{vertex} = [draw, circle]

\theoremstyle{definition}
\newtheorem{definition}{Definition}[section]
\newtheorem{remark}{Remark}
\newtheorem{lemma}{Lemma}
\newtheorem{claim}{Claim}
\newtheorem{conjecture}{Conjecture}
\newtheorem{construction}{Construction}
\newtheorem{theorem}{Theorem}
\newtheorem{example}{Example}
\newtheorem{hint}{Hint}
\setlength{\parindent}{0ex}

\rhead[Lecture Notes 5]{Lecture Notes 5}
\lhead[CS-E4340 Cryptography, Chris Brzuska]{CS-E4340 Cryptography, Chris Brzuska}

\newcommand{\M}[1]{\ensuremath{\text{\texttt{#1}}}}
\renewcommand{\O}[1]{\ensuremath{\mathsf{#1}}}
\newcommand{\pcvar}[1]{\ensuremath{\mathit{#1}}}
%\newcommand{\pcassert}{\ensuremath{\mathbf{assert}\;}} cb: already defined in the new version of cryptocode
\newcommand{\pckw}[1]{\highlightkeyword{#1}}
\newcommand{\pctype}[1]{\mathsf{#1}}
\newcommand{\ininterface}[1]{[#1\rightarrow]} %{I^{in}_{#1}}
\newcommand{\outinterface}[1]{[\rightarrow#1]} %{I^{out}_{#1}}
\newcommand{\m}{\pcvar{m}} %{I^{out}_{#1}}
\newcommand{\ver}{\O{ver}} %{I^{out}_{#1}}
\newcommand{\se}{\O{se}} %{I^{out}_{#1}}

\renewcommand{\enc}{\O{enc}} %{I^{out}_{#1}}
\renewcommand{\dec}{\O{dec}} %{I^{out}_{#1}}

\renewcommand{\mac}{\O{mac}}
\renewcommand{\circ}{\rightarrow}

\newcommand\defeq{\mathrel{\overset{\makebox[0pt]{\mbox{\normalfont\tiny\sffamily
def}}}{=}}}
\def\checkmark{\tikz\fill[scale=0.4](0,.35) -- (.25,0) -- (1,.7) --
(.25,.15) -- cycle;}

\title{Lecture 5: Symmetric Encryption}
\author{Chris Brzuska}
\date{\today}

\begin{document}
\paragraph{Remark} Lecture Video 5 partly covered the proof of \emph{PRF $\Rightarrow$ UNF-CMA-secure MAC}. The notes
for this proof are contained in Lecture Notes 4. The current lecture notes cover part II of Lecture Video 5 which
defines symmetric-key encryption schemes. We provide additional theorems and proofs in the current lecture notes which
were not contained in the Lecture Video 5. Concretely,
the current lecture notes contain
\begin{itemize}
\item the syntax of symmetric-key encryption schemes
\item confidentiality of symmetric-key encryption schemes, captured by a security notion which we call \emph{indistinguishability under chosen plaintext attacks (IND-CPA)}
\item confidentiality and integrity of symmetric-key encryption schemes, captured by the security notion of \emph{authenticated encryption} (AE).
\item Theorems (informal):
\begin{itemize}
\item A PRF in counter-mode (CTR-mode) yields IND-CPA-secure symmetric encryption.
\item A pseudorandom permutation (PRP) in CBC-mode yields IND-CPA-secure symmetric encryption.
\item AE-secure symmetric encryption schemes are also IND-CPA-secure. 
\item If the encryption algorithm of a symmetric encryption scheme is deterministic, then it is not IND-CPA-secure. 
\end{itemize}
\end{itemize}


\section{Syntax of symmetric-key encryption schemes}
A symmetric encryption scheme consists of two algorithms, an \emph{encryption algorithm}
and a \emph{decryption algorithm}. The encryption algorithm of a symmetric-key encryption
scheme takes a key $k$ and a message $m$, and encrypts them into a ciphertext $c$. The
encryption process is \emph{randomized}. This is crucial for security, as we will
see in Theorem~\ref{thm:deterministic}.
\begin{definition}[Syntax of symmetric encryption schemes]
  A \emph{symmetric encryption scheme} $\se$ consists of two probabilistic polynomial-time (PPT) algorithms
  \begin{align*}
	c \sample&\se.\enc(k,m) \\
	m\gets&\se.\dec(k,c)
  \end{align*} 
  which have to satisfy the correctness criterion
  \[\forall m\in\bin^* \probsub{k\sample\bin^n}{\se.\dec(k,\se.\enc(k,m))=m}=1\]
  i.e. when a ciphertext is created using the encryption algorithm, the decryption algorithm always returns the original message.
\end{definition}

\section{IND-CPA-security for symmetric-key encryption schemes}
Symmetric-encryption schemes should provide confidentiality. How do we define confidentiality?
A first thought might be that \emph{one-wayness} might be a good way to capture confidentiality.
However, as we saw in the first lecture, one-wayness is a very weak security notion, and our
symmetric encryption scheme might leak as much as half of the message if we use one-wayness as
 a security notion. Thus, one-wayness would be a terrible way to capture confidentiality.

Instead, what we will capture is the property that encryption should really not leak any information
except for the length of the message---it has to leak the length of the message, since information-theoretically,
longer messages have to be encrypted into longer ciphertexts. Now, if we cannot distinguish whether
the message $m$ was encrypted or the message $0^{\abs{m}}$, then the ciphertext carries no information
about $m$ (except for the length of $m$).

We have already stated what we want, namely, that encryptions of $m$ look like encryptions of $0^{\abs{m}}$.
But how do we choose $m$? We could define a game which chooses $m$ uniformly at random from all bitstrings
of some length. But this does not model real-life distributions very well. We could also parametrize our
system with a distribution over messages, but this is a cumbersome definition.

To resolve this issue, we will apply a trick which we already used in the context of message authentication
codes: We allow the \emph{adversary} to choose the message! This simplifies our model and additionally gives
us stronger properties (which we need if we recall the BEAST attack): Even if the adversary happens to be
able to influence the content of the message, security of the system does not break down.

It might be a little counter-intuitive to model confidentiality by an adversary who already knows the
message---but in fact, this only makes our property even stronger: Even though the adversary knows what
$m$ might be, it cannot distinguish between an encryption of $m$ and an encryption of $0^{\abs{m}}$.

\begin{definition}[IND-CPA]
	A symmetric encryption scheme $\se$ is IND-CPA-secure if the real game $\M{Gind\text{-}cpa}^0$ and the ideal game $\M{Gind\text{-}cpa}^1$ are computationally indistinguishable, that is, for all PPT adversaries $\adv$, the advantage
	\begin{align*}
	&\mathbf{Adv}_{\adv}^{
		\M{Gind\text{-}cpa}^0
		,\M{Gind\text{-}cpa}^1}
		(\lambda)\\
		&:=\abs{\prob{1=\adv\circ \M{Gind\text{-}cpa}^0}-\prob{1=\adv\circ \M{Gind\text{-}cpa}^1}}
	\end{align*}
	is negligible in $\lambda$.
  \begin{center}
		\begin{pchstack}
		  \begin{pcvstack}
					\underline{\underline{$\M{Gind\text{-}cpa}^0_\se$}}\\
				  \procedure{Parameters}{
					\lambda\text{:} \< \quad \text{sec. parameter}\\
					%b\text{:} \< \quad \text{ideal. bit}\; b=0\\
					%in\text{:} \< \quad \text{input length}\;\lambda\;\text{or}\;*\\
					%				out\text{:} \< \quad \text{output length}\,\lambda\,\text{or}\,*\\
				  \se\text{:} \< \quad \text{sym. enc. sch.} 
							  }
											\pcvspace
														  \procedure{Package State}{
				   k\text{:} \< \quad \text{key} \\
							  }
											\pcvspace
			  \procedure{$\O{ENC}(x)$}{
					\pcif k=\bot:\\
					\pcind k\sample\bin^\lambda\\
				\\
				 c\sample \se.\enc(k,x)\\
				\pcreturn c}
			\end{pcvstack}
				 \pchspace
		  \begin{pcvstack}		
					\underline{\underline{$\M{Gind\text{-}cpa}^1_\se$}}\\
				  \procedure{Parameters}{
					\lambda\text{:} \< \quad \text{sec. parameter}\\
					%b\text{:} \< \quad \text{ideal. bit}\; b=1\\
					%in\text{:} \< \quad \text{input length}\;\lambda\;\text{or}\;*\\
					%				out\text{:} \< \quad \text{output length}\,\lambda\,\text{or}\,*\\
				  \se\text{:} \< \quad \text{sym. enc. sch.} 
							  }
											\pcvspace
														  \procedure{Package State}{
				   k\text{:} \< \quad \text{key} \\
							  }
											\pcvspace
			  \procedure{$\O{ENC}(x)$}{
				  \pcif k=\bot:\\
					\pcind k\sample\bin^\lambda\\
					x'\gets 0^{\abs{x}}\\
				 c\sample \se.\enc(k,x')\\
				\pcreturn c}
			\end{pcvstack}
		\end{pchstack}
	\end{center}
\end{definition}


\section{AE-Security}
\emph{Authenticated encryption} security, in addition to confidentiality, also captures integrity and authenticity.
Namely, the adversary is also allowed to make \emph{decryption} queries and can also be successful merely by
\emph{forging} a ciphertext. This is encoded by having the ideal decryption oracle return $0$ for all fresh
ciphertexts, i.e., for all ciphertexts which did not come from the encryption oracle.


\begin{definition}[AE-security]
	A symmetric encryption scheme is AE-secure if the real game $\M{Gae}^0_\se$ and the ideal game $\M{Gae}^1_\se$ are computationally indistinguishable, that is, for all PPT adversaries $\adv$, the advantage
	\[\mathbf{Adv}_{\adv}^{\M{Gae}^0_\se,\M{Gae}^1_\se}(\lambda):=\abs{\prob{1=\adv\circ \M{Gae}^0_\se}-\prob{1=\adv\circ \M{Gae}^1_\se}}\]
	is negligible in $\lambda$.
		\begin{center}
			\begin{pchstack}
			  \begin{pcvstack}
						\underline{\underline{$\M{Gae}^0_\se$}}\\
					  \procedure{Parameters}{
						\lambda\text{:} \< \quad \text{sec. parameter}\\
						%b\text{:} \< \quad \text{ideal. bit}\; b=0\\
						%in\text{:} \< \quad \text{input length}\;\lambda\;\text{or}\;*\\
						%				out\text{:} \< \quad \text{output length}\,\lambda\,\text{or}\,*\\
					  \se\text{:} \< \quad \text{sym. enc. sch.} 
								  }
												\pcvspace
															  \procedure{Package State}{
					   k\text{:} \< \quad \text{key} \\
								  }
												\pcvspace
				  \procedure{$\O{ENC}(x)$}{
						\pcif k=\bot:\\
						\pcind k\sample\bin^\lambda\\
					\\
					 c\sample \se.\enc(k,x)\\
					\\
					\pcreturn c}
																\pcvspace
				  \procedure{$\O{DEC}(c)$}{
						\pcif k=\bot:\\
						\pcind k\sample\bin^\lambda\\
					  x\gets \se.\dec(k,c)\\
					\pcreturn x}
				\end{pcvstack}
					 \pchspace
			  \begin{pcvstack}
						\underline{\underline{$\M{Gae}^1_\se$}}\\
					  \procedure{Parameters}{
						\lambda\text{:} \< \quad \text{sec. parameter}\\
						%b\text{:} \< \quad \text{ideal. bit}\; b=1\\
						%in\text{:} \< \quad \text{input length}\;\lambda\;\text{or}\;*\\
						%				out\text{:} \< \quad \text{output length}\,\lambda\,\text{or}\,*\\
					  \se\text{:} \< \quad \text{sym. enc. sch.} 
								  }
												\pcvspace
															  \procedure{Package State}{
						k\text{:} \< \quad \text{key} \\
						T\text{:} \< \quad \text{table} 
								  }
												\pcvspace
				  \procedure{$\O{ENC}(x)$}{
						\pcif k=\bot:\\
						\pcind k\sample\bin^\lambda\\
						x'\gets 0^{\abs{x}}\\
					  c\sample \se.\enc(k,x')\\
					T[c]\gets x\\
					\pcreturn c}
					\pcvspace
				  \procedure{$\O{DEC}(c)$}{
					   x\gets T[c]\\
					\pcreturn x}
				\end{pcvstack}
			\end{pchstack}
		\end{center}
\end{definition}	


\section{Theorems}

We describe how to build secure symmetric encryption schemes. We start with IND-CPA security
and then turn to AE-security. Since PRFs and PRPs tend to operate on blocks, we will often
need an \emph{encoding} scheme which encodes messages into a multiple of the block length.
\begin{definition}[Encoding scheme]
We call two functions $\O{encode}_\lambda:\bin^*\rightarrow\bin^*$ and $\O{decode}_\lambda:\bin^*\rightarrow\bin^*$
an \emph{encoding scheme} if
\begin{itemize}
\item $\O{encode}_\lambda$ and $\O{decode}_\lambda$ are computable in time polynomial in $\lambda$ and the length of the input.
\item $\O{decode}_\lambda$ is the inverse of $\O{encode}_\lambda$, i.e., for all $x\in\bin^*$, $\O{decode}_\lambda$ can
recover $x$ from $\O{encode}_\lambda(x)$, that is, we have
      $\O{decode}_\lambda(\O{encode}_\lambda(x))=x$. In particular, $\O{encode}_\lambda$ is \emph{injective}, i.e., if $x\neq x'$, then $\O{encode}_\lambda(x)\neq\O{encode}_\lambda(x')$.
\item the length $\O{encode}_\lambda$ only depends on the length of the input, i.e., if $\abs{x}=\abs{x'}$, then 
$\abs{\O{encode}_\lambda(x)}=\abs{\O{encode}_\lambda(x')}$.
\item for all $x\in\bin^*$, $\abs{\O{encode}_\lambda(x)}$ is divisible by $\lambda$.
\end{itemize}
\end{definition}

\begin{theorem}[PRP in CBC is IND-CPA]
Let $(\O{encode}_\lambda,\O{decode}_\lambda)$ be an encoding scheme and let $f$ be a secure $(\lambda,\lambda)$-secure pseudorandom permutation. Then, the
following encryption scheme $\se_{\text{CBC-f}}$ is IND-CPA-secure.
\begin{center}
  \begin{pchstack}
  \procedure{$\se_{\text{CBC-f}}.\O{enc}(k,m)$}{
    \lambda\gets\abs{k}\\
    m'\gets\O{encode}_{\lambda}(m)\\
    \pcvar{nonce}\sample\bin^\lambda\\
    \pcvar{c}_0\gets\pcvar{nonce}\\
    \ell\gets \frac{\abs{m'}}{\lambda}\\
    \pcfor i=1..\ell\\
    \pcind x_i\gets m'_{(i-1)\lambda+1..i\lambda}\\
    \pcind \pcvar{c}_i\gets f(k,x_i\oplus c_{i-1})\\
    c\gets (c_0,..,c_\ell)\\
    \pcreturn c
  }
  \pchspace
  \procedure{$\se_{\text{CBC-f}}.\O{dec}(k,c)$}{
    \lambda\gets\abs{k}\\
    \ell\gets \frac{\abs{c}}{\lambda}-1\\
    \text{Parse }c_0,..,c_\ell\gets c\\
    \pcfor i=1..\ell\\
    \pcind x_i\gets c_{i-1}\oplus f_{\text{inv}}(k,c_i)\\
    m'\gets x_1||..||x_\ell\\
    m\gets \O{decode}_{\abs{k}}(m')\\
    \pcreturn m
  }
  \end{pchstack}
\end{center}
\end{theorem}

\begin{theorem}[PRF in CTR is IND-CPA]
Let $(\O{encode}_\lambda,\O{decode}_\lambda)$ be an encoding scheme and let $f$ be a secure $(\lambda,\lambda)$-secure pseudorandom function. Then, the encryption scheme $\se_{\text{CTR-f}}$ is IND-CPA-secure.
\begin{center}
\begin{pchstack}
\procedure{$\se_{\text{CTR-f}}.\O{enc}(k,m)$}{
\lambda\gets\abs{k}\\
\pcvar{nonce}\sample\bin^\lambda\\
m'\gets\O{encode}_\lambda(m)\\
\ell\gets  \frac{\abs{m'}}{\lambda}\\
\pcfor i=1..\ell\\
\pcind \pcvar{pad}_i\gets f(k,\pcvar{nonce}+i)\\
\pcvar{pad}'\gets\pcvar{pad}_1||..||\pcvar{pad}_\ell\\
\pcvar{pad}\gets\pcvar{pad}_{1..\abs{m}}\\
c\gets m \oplus \pcvar{pad}\\
\pcreturn (\pcvar{nonce},c)}
\pchspace
\procedure{$\se_{\text{CTR-f}}.\O{dec}(k,(\pcvar{nonce},c))$}{
\lambda\gets\abs{k}\\
\ell\gets \frac{\abs{c}}{\lambda}\\
\pcfor i=1..\ell\\
\pcind \pcvar{pad}_i\gets f(k,\pcvar{nonce}+i)\\
\pcvar{pad}'\gets\pcvar{pad}_1||..||\pcvar{pad}_\ell\\
\pcvar{pad}\gets\pcvar{pad}_{1..\abs{c}}\\
m'\gets c \oplus \pcvar{pad}\\
m\gets\O{decode}_\lambda(m')\\
\pcreturn m}
\end{pchstack}
\end{center}
\end{theorem}

\begin{theorem}[... is AE]
Will be added after Monday, October 10, 2022. %(to avoid harming the joy of researching the question).
\end{theorem}

\paragraph{Generic transformations on symmetric encryption schemes}
\begin{theorem}
If $\se$ is an IND-CPA secure encryption scheme, then $\se_1$ is an IND-CPA secure encryption scheme.

\begin{center}
\begin{pchstack}
        \procedure{$\se_1.\enc(k,m)$}{
          c'\sample\se.\enc(k,m)\\
          c \gets c'||1 \\
          \pcreturn c
        }
        \pcvspace
        \procedure{$\se_1.\dec(k,c)$}{
          c'\gets c[1..\abs{c}-1]\\
          m\gets\se.\dec(k,c')\\
          \pcreturn m
        }
\end{pchstack}
\end{center}
\end{theorem}

\begin{theorem}\label{thm:deterministic}
Let $\se$ be a symmetric encryption scheme such that $\se.\enc$ is deterministic.
Then $\se$ is not IND-CPA-secure.
\end{theorem}


\end{document}