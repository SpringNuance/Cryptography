\section{Equivalent Notions of Advantage}\label{app:calculus}
In this appendix, we prove the following claim:
\begin{claim}
It holds that
\begin{equation*}
2\cdot\tilde{\mathbf{Adv}}_\adv^{\M{G}^0,\M{G}^1}(\lambda)=\mathbf{Adv}_\adv^{\M{G}^0,\M{G}^1}(\lambda),
\end{equation*}
where
\[\mathbf{Adv}_\adv^{\M{G}^0,\M{G}^1}(\lambda):=\abs{\prob{1 = \adv \circ \M{G}^0} - \prob{1 = \adv \circ \M{G}^1}}\]
and
\[\tilde{\mathbf{Adv}}_\adv^{\M{G}^0,\M{G}^1}(\lambda):=\abs{\probsub{b\sample\bin}{b = \adv \circ \M{G}^b} - \frac{1}{2}}.\]
\end{claim}
\begin{proof}
\begin{align*}
\mathbf{Adv}_\adv^{\M{G}^0,\M{G}^1}(\lambda)\\
=&\abs{\prob{1 = \adv \circ \M{G}^0} - \prob{1 = \adv \circ \M{G}^1}}\\
=&\abs{\prob{1 = \adv \circ \M{G}^1}-\prob{1 = \adv \circ \M{G}^0}}\\
=&\abs{-\prob{1 = \adv \circ \M{G}^0} + \prob{1 = \adv \circ \M{G}^1}}\\
=&\abs{(1-\prob{1 = \adv \circ \M{G}^0}) + \prob{1 = \adv \circ \M{G}^1} - 1}\\
=&2\cdot\abs{\frac{1}{2}\cdot (1-\prob{1 = \adv \circ \M{G}^0}) + \frac{1}{2}\cdot \prob{1 = \adv \circ \M{G}^1} - \frac{1}{2}}\\
=&2\cdot\abs{\frac{1}{2}\cdot \prob{0 = \adv \circ \M{G}^0} + \frac{1}{2}\cdot \prob{1 = \adv \circ \M{G}^1} - \frac{1}{2}}\\
=&2\cdot\abs{\probsub{b\sample\bin}{b = \adv \circ \M{G}^b} - \frac{1}{2}}\\
=&2\cdot\tilde{\mathbf{Adv}}_\adv^{\M{G}^0,\M{G}^1}(\lambda)\\
\end{align*}
\end{proof}
