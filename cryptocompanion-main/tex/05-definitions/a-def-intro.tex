The previous section focused on foundations, attempting to give a ground up explanation of how we look at cryptography. This is interesting, and necessary for thorough understanding. Another important thing in cryptography are proofs: they deepen our understanding, and enable us to convince others that statements are true. To write proofs, we need to be rather precise: Firstly, we need to be precise about the \emph{statements} that we try to prove (this is what we need definitions for), and secondly, we need to be precise about which \emph{steps} we are allowed to make in proofs. The aim of this section is to enable precise communication by introducing a language for talking about cryptographic security, and explaining how statements are written in that language.

We start off by discussing some mathematical notation that forms the core of the language we use to talk about cryptography. Building on that, we define the pseudocode we use to describe algorithms, and the ways in which we combine pieces of pseudocode to describe larger systems. With these definitions in place, we move on to definitions related to security: how exactly is the adversary modeled, and when exactly can we call a system secure. 

This is a section you should at least skim; otherwise it's likely you will be confused when reading statements/proofs or trying to write them yourself. If you don't understand some phrase in a definition or a proof, this is the place you should come. This section is useful for looking up technical definitions---for additional conceptual discussion, see Section~\ref{section:foundations}.