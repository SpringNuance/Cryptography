\newif\iffull\fullfalse% do not change flags here
\newif\ifdraft\draftfalse% do not change flags here

\documentclass[envcountsame,runningheads,notitlepage]{../llncs}
\pagestyle{plain}

\usepackage[letterpaper,hmargin=3.5cm,vmargin=3cm]{geometry}
\usepackage[T1]{fontenc}
\usepackage{lmodern}
\usepackage{xifthen}
\usepackage{amsmath}
\usepackage{amsfonts}
\usepackage{amssymb}
\usepackage{lscape}
%\usepackage{amsthm}
\usepackage{xspace}
\usepackage{enumerate}
\usepackage{tikz}
\usepackage{hyperref}
\usepackage{color,soul}
\usepackage{tikz}
\usetikzlibrary{shapes,arrows, calc}
\usepackage{graphicx}
\usepackage{array}
\usepackage{subcaption}
\usepackage[probability,adversary,sets,notions,operators,complexity,keys,primitives,asymptotics,advantage]{cryptocode}
\usepackage{savesym}
\usepackage{relsize}
\savesymbol{cb}
\usepackage{combelow}
\usepackage[normalem]{ulem}
\usepackage{wrapfig}
\newtheorem{crossed}{Esitehtävä}
\usepackage{enumitem}

\let\proof\relax
\let\endproof\relax
\usepackage{amsthm}

\theoremstyle{definition}
\newtheorem{graded}[crossed]{Exercise}
\newtheorem{challenge}[crossed]{Challenge exercise}

% Colors
\newcommand\maybecolor[1]{\color{#1}}
\definecolor{lightblue}{rgb}{0.2,0.2,1.0}
\definecolor{dkred}{rgb}{0.5, 0.0,0.0}
\definecolor{lightgrey}{rgb}{0.8,0.8,0.8}
\definecolor{dkblue}{rgb}{0,0.1,0.5}
\definecolor{dkgreen}{rgb}{0,0.4,0}
\definecolor{dkred}{rgb}{0.6,0,0}
\definecolor{linkColor}{rgb}{0,0,0.5}
\definecolor{lightgray}{rgb}{.9,.9,.9}
\definecolor{darkgray}{rgb}{.4,.4,.4}
\definecolor{purple}{rgb}{0.65, 0.12, 0.82}

\ifdraft
\newcommand{\fullcolor}{\color{blue}}
\else
\newcommand{\fullcolor}{\color{black}}
\fi
\newcommand{\blk}{\color{black}}
\newcommand{\dkg}{\color{dkgreen}}

% Variable and procedures names
\renewcommand{\O}[1]{\ensuremath{\mathsf{#1}}}
\newcommand{\V}[1]{\ensuremath{\mathit{#1}}}
\newcommand{\M}[1]{\ensuremath{\text{\tt#1}}}
\def \G {\mathcal{G}}
\newcommand{\hLen}{n}




%\newcommand{\markulf}[1]{{\color{orange}{MK: #1}}}
\newcommand{\mk}[1]{{\color{orange}{mk: #1}}}


% WIDEBAR

\makeatletter
\let\save@mathaccent\mathaccent
\newcommand*\if@single[3]{%
  \setbox0\hbox{${\mathaccent"0362{#1}}^H$}%
  \setbox2\hbox{${\mathaccent"0362{\kern0pt#1}}^H$}%
  \ifdim\ht0=\ht2 #3\else #2\fi
  }
%The bar will be moved to the right by a half of \macc@kerna, which is computed by amsmath:
\newcommand*\rel@kern[1]{\kern#1\dimexpr\macc@kerna}
%If there's a superscript following the bar, then no negative kern may follow the bar;
%an additional {} makes sure that the superscript is high enough in this case:
\newcommand*\widebar[1]{\@ifnextchar^{{\wide@bar{#1}{0}}}{\wide@bar{#1}{1}}}
%Use a separate algorithm for single symbols:
\newcommand*\wide@bar[2]{\if@single{#1}{\wide@bar@{#1}{#2}{1}}{\wide@bar@{#1}{#2}{2}}}
\newcommand*\wide@bar@[3]{%
  \begingroup
  \def\mathaccent##1##2{%
%Enable nesting of accents:
    \let\mathaccent\save@mathaccent
%If there's more than a single symbol, use the first character instead (see below):
    \if#32 \let\macc@nucleus\first@char \fi
%Determine the italic correction:
    \setbox\z@\hbox{$\macc@style{\macc@nucleus}_{}$}%
    \setbox\tw@\hbox{$\macc@style{\macc@nucleus}{}_{}$}%
    \dimen@\wd\tw@
    \advance\dimen@-\wd\z@
%Now \dimen@ is the italic correction of the symbol.
    \divide\dimen@ 3
    \@tempdima\wd\tw@
    \advance\@tempdima-\scriptspace
%Now \@tempdima is the width of the symbol.
    \divide\@tempdima 10
    \advance\dimen@-\@tempdima
%Now \dimen@ = (italic correction / 3) - (Breite / 10)
    \ifdim\dimen@>\z@ \dimen@0pt\fi
%The bar will be shortened in the case \dimen@<0 !
    \rel@kern{0.6}\kern-\dimen@
    \if#31
      \overline{\rel@kern{-0.6}\kern\dimen@\macc@nucleus\rel@kern{0.4}\kern\dimen@}%
      \advance\dimen@0.4\dimexpr\macc@kerna
%Place the combined final kern (-\dimen@) if it is >0 or if a superscript follows:
      \let\final@kern#2%
      \ifdim\dimen@<\z@ \let\final@kern1\fi
      \if\final@kern1 \kern-\dimen@\fi
    \else
      \overline{\rel@kern{-0.6}\kern\dimen@#1}%
    \fi
  }%
  \macc@depth\@ne
  \let\math@bgroup\@empty \let\math@egroup\macc@set@skewchar
  \mathsurround\z@ \frozen@everymath{\mathgroup\macc@group\relax}%
  \macc@set@skewchar\relax
  \let\mathaccentV\macc@nested@a
%The following initialises \macc@kerna and calls \mathaccent:
  \if#31
    \macc@nested@a\relax111{#1}%
  \else
%If the argument consists of more than one symbol, and if the first token is
%a letter, use that letter for the computations:
    \def\gobble@till@marker##1\endmarker{}%
    \futurelet\first@char\gobble@till@marker#1\endmarker
    \ifcat\noexpand\first@char A\else
      \def\first@char{}%
    \fi
    \macc@nested@a\relax111{\first@char}%
  \fi
  \endgroup
}
\makeatother


%%%%%%%%%%%%%%%%%%%%%%%%%%%%%%%%%%%%%
%     FONTS AND EDITING
%%%%%%%%%%%%%%%%%%%%%%%%%%%%%%%%%%%%%
%%%%%%%%%%%%%%%%%%%%%%%%%%%%%%%%%%%%%

\newcommand{\highlight}[1]{\colorbox{red!20}{$\displaystyle#1$}}
\definecolor{purple}{RGB}{139,0,139}
\newcommand{\ben}[1]{{\color{purple}{BD: #1}}}
\definecolor{darkgreen}{RGB}{0,128,0}
\newcommand{\chris}[1]{{\color{darkgreen}{CB: #1}}}
\definecolor{orange}{RGB}{210,105,30}
\newcommand{\geoffroy}[1]{{\color{orange}{GC: #1}}}
\newcommand{\todo}[1]{{\color{red}{TODO: #1}}}

\definecolor{darkblue}{RGB}{0,0,128}
\definecolor{midnightblue}{RGB}{0,0,210}
\definecolor{darkred}{RGB}{128,0,0}
\definecolor{figred}{RGB}{230,0,26}
\definecolor{figgreen}{RGB}{153,192,0}
\definecolor{figorange}{RGB}{245,163,0}
\definecolor{figblue}{RGB}{0,104,157}
\definecolor{figpurple}{RGB}{149,17,105}




%\theoremstyle{plain}
%\newtheorem{theorem}{Theorem}
%\newtheorem{proposition}{Proposition}
%\newtheorem{lemma}{Lemma}
%\newtheorem*{corollary}{Corollary}

%\newtheorem{definition}{Definition}
%\newtheorem{conjecture}{Conjecture}
%\newtheorem{algorithm}{Algorithm}
\newtheorem{construction}{Construction}
%\newtheorem{claim}{Claim}

%\theoremstyle{remark}
%\newtheorem*{remark}{Remark}
%\newtheorem*{note}{Note}
%\newtheorem{case}{Case}


%%%%%%%%%%%%%%%%%%%%%%%%%%%%%%%%%%%%%
%         STYLES
%%%%%%%%%%%%%%%%%%%%%%%%%%%%%%%%%%%%%

\newcommand{\R}{\mathbb{R}} \newcommand{\N}{\mathbb{N}}
\newcommand{\Z}{\mathbb{Z}} \newcommand{\Q}{\mathbb{Q}}
\newcommand{\card}[1]{\abs{#1}}
%\theoremstyle{definition} \newtheorem*{definition}{Definition}
%\newtheorem*{remark}{Remark} \newtheorem*{lemma}{Lemma}
%\newtheorem*{conjecture}{Conjecture} \newtheorem*{hint}{Hint}
%\newtheorem{claim}{Claim}
\newcommand{\cnt}{\textsf{ctr}}
\renewcommand{\H}{\textsf{H}}
\newcommand{\RO}{\textsf{RO}}
\newcommand{\RPO}{\textsf{RPO}}
\renewcommand{\O}[1]{\ensuremath{\mathsf{#1}}}
\newcommand{\E}{\textsf{E}}
\newcommand{\ininterface}[1]{{\mathsf{in}(#1)}} %{I^{in}_{#1}}
\newcommand{\outinterface}[1]{{\mathsf{out}(#1)}} %{I^{out}_{#1}}
\newcommand{\StyleModel}[1]{\V{{#1}}}

\newcommand{\kgen}{\StyleModel{kgen}}
\newcommand{\m}{\StyleModel{m}}
\newcommand{\se}{\StyleModel{se}}
\renewcommand{\enc}{\StyleModel{enc}}
\renewcommand{\dec}{\StyleModel{dec}}
%\newcommand{\prp}{\StyleModel{prp}}
\newcommand{\nonce}{\StyleModel{nonce}}
\newcommand{\ac}{\StyleModel{ac}}
\newcommand{\ch}{\StyleModel{ch}}
\renewcommand{\mac}{\StyleModel{mac}}
\newcommand{\ver}{\StyleModel{ver}}
\renewcommand{\verify}{\StyleModel{ver}}
\newcommand{\receive}{\StyleModel{receive}}
\newcommand{\send}{\StyleModel{send}}
\newcommand{\unfcma}{\M{UNF-CMA}}
\newcommand{\unfcmaac}{\M{UNF-CMA-AC}}
\newcommand{\flunfcma}{\M{FL-UNF-CMA}}
%\newcommand{\pcassert}{\mathbf{assert}\;} If you get an error saying that pcassert is not defined yet, please update the cryptocode package.
%\specialcomment{solution}{\textbf{Solution:}\;}{}
\newboolean{ifsolution}
\setboolean{ifsolution}{true}

\usepackage[breakable]{tcolorbox}

% if exercises only:
\newcommand{\sol}[1]{}

% if solution included:
%\newcommand{\sol}[1]{\begin{tcolorbox}[breakable]\solution{#1}\end{tcolorbox}}

\title{CS-E4340 Cryptography: Exercise Sheet 1\\[0.5\baselineskip]
\large ---One-Way Functions, Algorithms \& Probabilities---}
\author{}\institute{}%Submission deadline: September 12, 2022, 11:30, via MyCourses}\institute{}
%\author{Christopher Brzuska, Pihla Karanko}
%\institute{
%			 Aalto University, Finland
%}
%\date{\today}
\date{}

\begin{document}

\maketitle

\noindent
\textbf{Submission deadline: September 12, 2022, 11:30, via MyCourses}

\medskip
\noindent
Each exercise can give up to two participation points, 2 for a mostly correct solution and 1 point for a good attempt. Overall, the exercise sheet gives at most 4 participation points. We encourage to \textbf{choose} exercises which seem \textbf{interesting} and/or adequately challenging to you. %In particular, you can skip some exercises, if you like. Your teaching assistant will carefully study your ideas and provide helpful suggestions. 


\medskip
\noindent
Exercise Sheet 1 is intended to help...
\begin{itemize}
\item[(a)] ...understand the \emph{definition} of One-Way Functions (OWFs) and gain \emph{intuition} for OWFs.
\item[(b)] ...familiarize yourself with the idea of generic counterexamples.
\item[(c)] ...familiarize yourself with security experiments.
\item[(d)] ...practice thinking about probabilities.
\item[(e)] ...practice the use of pseudocode.
\end{itemize}

\begin{description}
\item[Exercise 1] aims to help understand probabilities. 
\item[Exercise 2] helps understand the notion of one-wayness and is the \underline{\textbf{most important}} exercise on this sheet. We warmly encourage it.
\item[Exercise 3] gives an opportunity to practice writing \emph{inverters}\footnote{For OWFs, the term \emph{adversary} and \emph{inverter} are synonymous, because an adversary against a OWF tries to invert.} for some easy-to-invert functions, i.e. bad one-way function candidates.
\item[Ex. 4 \& Ex. 5] are advanced exercises, where you are asked to provide an attack on a generic one-way function counterexample and analyze the probability of the attack.
\end{description}



\begin{graded}[Probability and Pseudocode] \emph{2 points}
\begin{itemize}
\item[(a)] You roll three (six-sided) dice $D_1$, $D_2$ and $D_3$. There are $6\cdot 6\cdot 6=216$ possible combinations of the results $(D_1,D_2,D_3)$. For how many of the results is it true that $D_1+D_2+D_3=17$ ? Divide this number by $216$ and determine: What is $\probsub{D_1,D_2,D_3}{D_1+D_2+D_3=17}$, i.e., the probability that the sum $D_1+D_2+D_3$ is equal to $17$?

\item[(b)] Define the function $f$ and attacker $\adv$ as

\begin{center}
\begin{pchstack}
\procedure{$f(x)$}{
y \leftarrow x\oplus 1^{\abs{x}} \\
\pcreturn y
}
\pchspace
\procedure{$\adv(y,1^{\lambda})$}{
z \leftarrow y \oplus 1^{\lambda} \\
\pcreturn z
}
\end{pchstack}
\end{center}
and show that it holds that $\prob{\mathsf{Exp}_{f,\adv}^{\mathsf{OW}}(1^\lambda) = 1}=1$. The above $\oplus$ means bitwise XOR operation. For more notation, we refer to the crypto companion~\url{https://github.com/cryptocompanion/cryptocompanion}. Recall that the experiment $\mathsf{Exp}_{f,\adv}^{\mathsf{OW}}(1^\lambda)$ is defined as:
\begin{center}
	  \procedure{$\mathsf{Exp}_{f,\adv}^{\mathsf{OW}}(1^\lambda)$}{
		x\sample\bin^\lambda\\
		y\gets f(x)\\
		x'\sample\adv(y, 1^\lambda)\\
				\pcif \abs{x'}\neq\lambda \pcthen\\
		\pcind \pcreturn 0\\
		\pcif f(x')=y \pcthen\\
		\pcind \pcreturn 1\\
		\pcreturn 0}
\end{center}

\end{itemize}
\end{graded} % A bad scheme

\begin{graded}[One-Way Functions] \emph{2 points}
Assume the existence of a length-preserving one-way function\footnote{A function $f:\bin^*\rightarrow\bin^*$ is \emph{length-preserving} if for all $x\in\bin^*$, it holds that $\abs{f(x)}=\abs{x}$.}.
Say for each of the following statements whether you believe they are true or false and provide your intuition. You are not expected to \emph{know} the answer to these questions, i.e., reasoning suffices (for 2 points) even if not all answers are correct. $||$ denotes concatenation of strings.

\medskip
\noindent
\textbf{Hint.} recall from the lecture that if $f$ is one-way function, then $g^f_{leak-r}(x_l||x_r) := f(x_l)||x_r$ and $g^f_{app-zer}(x) := f(x)||0^\lambda$ are also one-way functions, where $|x_l|=|x_r|$. You can use the examples from the lecture and Section 4 of the crypto companion without justifying them.
\begin{enumerate}[label=(\alph*)]
\item \label{itm:f||g} For all length-preserving one-way functions $f$ and $g$, the following function $h$ is a one-way function: $h(x):=f(x)||g(x)$.
\item \label{itm:f||b} For all one-way functions $f$ and all polynomially computable functions $b$ with one bit output, the following function $h$ is a one-way function: $h(x):=f(x)||b(x)$.
\item \label{itm:g(f)} For all length-preserving one-way functions $f$ and $g$, the following function $h$ is a one-way function: $h(x):=g(f(x))$.
\item \label{itm:f no last half} For all length-preserving one-way functions $f$, the following function $h$ is a one-way function: $h(x):=f(x)_{1...\ceil{|x|/2}}$. I.e., $h$ returns all bits that $f$ returns, except for half of the bits (rounded up).
\item \label{itm:f no last bit} (\emph{Advanced}) For all length-preserving one-way functions $g$, the following function $h$ is a one-way function: $h(x):=g(x)_{1...|x|-1}$. I.e., $h$ returns all bits that $g$ returns, except for the last bit.
\item \label{itm:f xor g} For all length-preserving one-way functions $f, g$, the following function $h$ is a one-way function: $h(x):=f(x) \oplus g(x)$. I.e., $h$ is the bitwise XOR of two OWFs.
\item \label{itm:1 bit} There exists a one-way function $h$ with 1 bit output, i.e., for all $x\in\bin^*$, $|h(x)|=1$.
\item \label{itm:f||x xor g(f)} For all length-preserving one-way functions $f, g$, the following function $h$ is a one-way function: $h(x) := f(x) || (x \oplus g(f(x)))$. I.e., $h$ first applies $f$ to $x$ and then $g$ to $x$, then xors the result with $x$, this is the second half of the function $h$, and the first half of the function $h$ is just $f(x)$.
\item \label{itm:f(1)} For all length-preserving one-way functions $f$, the following function $h$ is a one-way function: $h(x) := f(1^{\abs{x}})$
\item \label{itm:f if first bit} For all length-preserving one-way functions $f$, the following function $h$ is a one-way function:
\begin{align*}
h(d||x) &:= \begin{cases}
    0 || f(x) &  \text{if } d=0\\
    1 || x &               \text{if } d=1.
\end{cases}
\end{align*}
where $d$ is of length 1 (first bit of the input).
\end{enumerate}
\end{graded}


\begin{graded}[Constructing Inverter] \emph{2 points}
Choose \emph{one} out of (h), (i) from Exercise 2, give an efficient inverter and argue that
the inversion probability is $1$. Alternatively, you can also choose \emph{one} out of $(j)$ or $(g)$ and give an inverter which
inverts with probability $\tfrac{1}{2}$.
\end{graded}

\begin{graded}[Attack a OWF-Candidate] \emph{2 points}
Choose one of the constructions $h$ in Exercise 2 (a)-(e) that is not one-way. Argue why it is not OWF, that is, provide an inverter and argue why the inverter is efficient (intuitive argument is enough).
\end{graded}

\begin{graded}[Analyze the Attacker] \emph{2 points}
What is the inversion probability of your inverter from the previous exercise? Justify your answer. Is the probability non-negligible?
\textbf{Hint:} Since we haven't discussed the definition of \emph{non-negligible}\footnote{A negligible function is a function tends to zero faster than any inverse polynomial as $\lambda$ tends to infinity. A non-negligible function is a function which is not negligible. See Definition 2.2 in the \emph{crypto companion}~\url{https://github.com/cryptocompanion/cryptocompanion}.} in the lecture, you can either look it up in the crypto companion, or simply argue that your inverter inverts with constant probability, e.g., $\tfrac{1}{10}$.
\end{graded}

\end{document}
%%% Local Variables:
%%% mode: latex
%%% TeX-master: t
%%% End:
