\documentclass[10pt,twoside]{article}
\setlength{\oddsidemargin}{0 in}
\setlength{\evensidemargin}{0 in}
\setlength{\topmargin}{-0.6 in}
\setlength{\textwidth}{6.5 in}
\setlength{\textheight}{8.5 in}
\setlength{\headsep}{0.75 in}
\setlength{\parindent}{0 in}
\setlength{\parskip}{0.1 in}

\newif\ifsolution
%\solutiontrue
 \solutionfalse
\newcommand{\inputsol}[1]{\ifsolution\begin{tcolorbox}[breakable]\input{sol/#1}\end{tcolorbox}\fi}

\usepackage{amsmath,amsfonts,graphicx,amssymb,amsthm}
\usepackage[
	lambda,
	operators,
	advantage,
	sets,
	adversary,
	landau,
	probability,
	notions,
	logic,
	ff,
	mm,
	primitives,
	events,
	complexity,
	asymptotics,
	keys]{cryptocode}
\usepackage[style=alphabetic]{biblatex}
\usepackage{url}
\usepackage[bookmarksdepth=2]{hyperref}
\usepackage[capitalize]{cleveref}
\hypersetup{
	colorlinks,
	linkcolor={red!50!black},
	citecolor={blue!50!black},
	urlcolor={blue!80!black}
}
\usepackage{xifthen}
\usepackage[inline]{enumitem}
\usepackage[breakable]{tcolorbox}

\newcounter{lecnum}
\renewcommand{\thepage}{\thelecnum-\arabic{page}}
\renewcommand{\thesection}{\thelecnum.\arabic{section}}
\renewcommand{\theequation}{\thelecnum.\arabic{equation}}
\renewcommand{\thefigure}{\thelecnum.\arabic{figure}}
\renewcommand{\thetable}{\thelecnum.\arabic{table}}

\newcommand{\coursecode}{CS-E4340}
\newcommand{\coursename}{Cryptography}
\newcommand{\courseterm}{I-II 2022/2023}
\newcommand{\lecturer}{Russell W.F. Lai}

\newcommand{\lecture}[3]{
	\pagestyle{myheadings}
	\thispagestyle{plain}
	\newpage
	\setcounter{lecnum}{#1}
	\setcounter{page}{1}
	\noindent
	\begin{center}
		\framebox{
			\vbox{\vspace{2mm}
				\hbox to 6.28in { {\bf \coursecode: \coursename \hfill \courseterm} }
				\vspace{4mm}
				\hbox to 6.28in { {\Large \hfill Lecture #1: #2  \hfill} }
				\vspace{2mm}
				\hbox to 1in { {\it Lecturer: #3} }
				\vspace{2mm}}
		}
	\end{center}
	\markboth{\coursecode (\courseterm) Lecture #1: #2}{\coursecode (\courseterm) Lecture #1: #2}
	\vspace*{4mm}
}

\newcommand{\tutorial}[3]{
   \pagestyle{myheadings}
   \thispagestyle{plain}
   \newpage
   \setcounter{lecnum}{#1}
   \setcounter{page}{1}
   \noindent
   \begin{center}
   \framebox{
      \vbox{\vspace{2mm}
      \hbox to 6.28in { {\bf \coursecode: \coursename \hfill \courseterm} }
      \vspace{4mm}
      \hbox to 6.28in { {\Large \hfill Tutorial #1: #2  \hfill} }
      \vspace{2mm}
      \hbox to 1in { {\it Tutor: #3} }
      \vspace{2mm}}
   }
   \end{center}
   \markboth{\coursecode (\courseterm) Tutorial #1: #2}{\coursecode (\courseterm) Tutorial #1: #2}
   \vspace*{4mm}
}

\newcommand{\assignment}[3]{
   \pagestyle{myheadings}
   \thispagestyle{plain}
   \newpage
   \setcounter{lecnum}{#1}
   \setcounter{page}{1}
   \noindent
   \begin{center}
   \framebox{
      \vbox{\vspace{2mm}
      \hbox to 6.28in { {\bf \coursecode: \coursename \hfill \courseterm} }
      \vspace{4mm}
      \hbox to 6.28in { {\Large \hfill Assignment #1: #2  \hfill} }
      \vspace{2mm}
      \hbox to 1in { {\it Deadline: #3} }
      \vspace{2mm}}
   }
   \end{center}
   \markboth{\coursecode (\courseterm) Assignment #1: #2}{\coursecode (\courseterm) Assignment #1: #2}
   \vspace*{4mm}
}

\newcommand{\exercise}[3]{
   \pagestyle{myheadings}
   \thispagestyle{plain}
   \newpage
   \setcounter{lecnum}{#1}
   \setcounter{page}{1}
   \noindent
   \begin{center}
   \framebox{
      \vbox{\vspace{2mm}
      \hbox to 6.28in { {\bf \coursecode: \coursename \hfill \courseterm} }
      \vspace{4mm}
      \hbox to 6.28in { {\Large \hfill Exercise #1: #2  \hfill} }
      \vspace{2mm}
      \hbox to 1in { {\it Deadline: #3} }
      \vspace{2mm}}
   }
   \end{center}
   \markboth{\coursecode (\courseterm) Exercise #1: #2}{\coursecode (\courseterm) Exercise #1: #2}
   \vspace*{4mm}
}

\def\beginrefs{\begin{list}%
        {[\arabic{equation}]}{\usecounter{equation}
         \setlength{\leftmargin}{2.0truecm}\setlength{\labelsep}{0.4truecm}%
         \setlength{\labelwidth}{1.6truecm}}}
\def\endrefs{\end{list}}
\def\bibentry#1{\item[\hbox{[#1]}]}

%Use this command for a figure; it puts a figure in wherever you want it.
%usage: \fig{NUMBER}{SPACE-IN-INCHES}{CAPTION}
\newcommand{\fig}[3]{
			\vspace{#2}
			\begin{center}
			Figure \thelecnum.#1:~#3
			\end{center}
	}

% Use these for theorems, lemmas, proofs, etc.
\theoremstyle{remark}
\newtheorem{theorem}{Theorem}[lecnum]
\newtheorem{lemma}[theorem]{Lemma}
\newtheorem{proposition}[theorem]{Proposition}
\newtheorem{claim}[theorem]{Claim}
\newtheorem{corollary}[theorem]{Corollary}
\newtheorem{definition}[theorem]{Definition}
\newtheorem*{definition*}{Definition}
\newtheorem{example}[theorem]{Example}
\newtheorem{remark}[theorem]{Remark}
\newtheorem{question}{Question}
\newtheorem{fact}{Fact}
% \newenvironment{proof}{{\bf Proof:}}{\hfill\rule{2mm}{2mm}}

\newcommand{\msg}{\mathsf{msg}}
\newcommand{\ctxt}{\mathsf{ctxt}}
\newcommand{\cX}{\mathcal{X}}
\newcommand{\bits}[1][]{\{0,1\}^{#1}}
\renewcommand{\vec}[1]{\mathbf{#1}}
\newcommand{\mat}[1]{\mathbf{#1}}
\newcommand{\inner}[2]{\langle #1, #2 \rangle}
\newcommand{\transpose}{\mathtt{T}}
\newcommand{\round}[1]{\lfloor #1 \rceil}
\newcommand{\SIS}{\mathsf{SIS}}
\newcommand{\LWE}{\mathsf{LWE}}
\newcommand{\ring}{\mathcal{R}}
\newcommand{\Hyb}{\mathsf{Hyb}}
\newcommand{\pubparam}{\mathsf{pp}}

\begin{document}
%\lecture{**LECTURE-NUMBER**}{**TOPIC**}{**LECTURER**}
%\tutorial{**TUTORIAL-NUMBER**}{**TOPIC**}{**TUTOR**}
%\assignment{**ASSIGNMENT-NUMBER**}{**TOPIC**}{**DEADLINE**}
%\exercise{**EXERCISE-NUMBER**}{**TOPIC**}{**DEADLINE**}
\exercise{7}{Introduction to Lattice-based Cryptography}{11:30 on November 7, 2022 via MyCourses as a single pdf file}


\begin{abstract}
    This exercise is designed to help students to ...
    \begin{itemize}
        \item understand modular arithmetic,
        \item understand the short integer solution (SIS) and learning with errors (LWE) assumptions,
        \item understand the leftover hash lemma in the lattice setting,
        \item be able to apply the above definitions and lemma to prove security of cryptographic schemes, and
        \item learn the notion of hiding and binding commitment.  
    \end{itemize} 
\end{abstract}

\begin{question}[Modular Arithmetic, Dual-Regev Encryption, and Linear Homomorphism] \quad \\

\textbf{Answer Part (a) and choose between answering either Part (b) or Part (c).}
\begin{enumerate}[label=(\alph*)]
    \item Consider $\ZZ_{13}$ (integers with arithmetic modulo $13$) represented by $\set{-6,\ldots,-1,0,1,\ldots,6}$. Calculate the following (writing down just the answer):
    \begin{enumerate*}[label=(\roman*)]
        \item $4+5 \bmod 13$,
        \item $-5 \times 2 \bmod 13$, and
        \item $6^{-1} \bmod 13$, i.e. the element $x \in \ZZ_{13}$ such that $6x = 1$.
    \end{enumerate*} 
    \item Let $n,m, \log p, \log q \in \poly$ with $p < q$ and $\chi$ be the uniform distribution over $\ZZ_\beta$ for some $\log \beta \in \poly$ with $\beta < q$. 
    In the following, we recall a slight generalisation of the dual-Regev encryption scheme with message space $\ZZ_p$:
    \begin{pchstack}[center,boxed]
        
        \procedure[]{$\kgen(\secparam)$}
        {
            \mat{A} \sample \ZZ_q^{n \times m} \\
            \vec{u} \sample \chi^{m} \\
            \vec{v} \coloneqq \mat{A} \cdot \vec{u} \bmod q \\
            \pk \coloneqq (\mat{A}, \vec{v}) \\
            \sk \coloneqq \vec{u} \\
            \pcreturn (\pk, \sk)
        }

        \pchspace

        \procedure[]{$\enc(\pk, x \in \ZZ_p)$}
        {
    				\mathsf{Parse}\;(\mat{A}, \vec{v}) \gets\pk\\
            \vec{s} \sample \ZZ_q^n \\
            \vec{e}_0 \sample \chi^{m} \\
            e_1 \sample \chi \\
            \vec{c}_0^\transpose \coloneqq \vec{s}^\transpose \cdot \mat{A} + \vec{e}_0^\transpose \bmod q \\
            c_1 \coloneqq \vec{s}^\transpose \cdot \vec{v} + e_1 + \floor{q/p} \cdot x \bmod q \\
            \ctxt \coloneqq (\vec{c}_0, c_1) \\
            \pcreturn \ctxt
        }

        \pchspace

        \procedure[]{$\dec(\sk, \ctxt)$}
        {   \mathsf{Parse}\;(\vec{c}_0, c_1) \gets\ctxt\\
				    \vec{u}\gets\sk\\
            \bar{x} \coloneqq c_1 - \vec{c}_0^\transpose \cdot \vec{u} \bmod q \\
            \pcreturn \left \lfloor \frac{p}{q} \cdot \bar{x} \right \rceil \\
            \pccomment{rounding to nearest integer} 
        }

    \end{pchstack}

    \begin{enumerate}[label=(\roman*)]
        \item Show that the scheme is correct when $q > m \cdot p \cdot \beta^2$ and $m \cdot \beta^2 \geq 2 \beta + 2$. 
        You can use the fact that for any $x,y \in \ZZ$ we have $\abs{x+y} \leq \abs{x} + \abs{y}$ and $\abs{x \cdot y} \leq \abs{x} \cdot \abs{y}$. 
        Furthermore, you may want to use the fact that $\abs{\frac{p}{q} \cdot \floor{\frac{q}{p}} - 1} \leq \frac{1}{q}$.
        [Hint: Note that decryption is correct when $\abs{\frac{p}{q} \cdot \bar{x} - x} < \frac{1}{2}$. Read the proof of correctness of the (primal-)Regev encryption scheme in the lecture notes.]
        \item Let $m \geq n \cdot \log_\beta q + \omega(\log n)$. 
        Prove via a reduction that the scheme is IND-CPA-secure under the $\LWE_{n,m+1,q,\chi}$ assumption. 
        [Hint: Read the proof of IND-CPA-security of the (primal-)Regev encryption scheme in the lecture notes. Follow the level of details of the lecture notes. The level of detail of the answer should be on a similar level as the lecture notes.]
    \end{enumerate} 

    \item In this question, we study the linearly homomorphic property of the dual-Regev encryption scheme, which is useful for understanding Lecture 9.
    You may assume that $m \cdot \beta^2 \geq 2 \beta + 2$. 
    [Hint: Read hint of Question 1 (b) (i)]
    \begin{enumerate}[label=(\roman*)]
        \item Let $(\pk, \sk) \in \kgen(\secparam)$, 
        $x, x' \in \ZZ_p$, 
        $\ctxt \coloneqq (\vec{c}_0, c_1) \in \enc(\pk, x)$, and 
        $\ctxt' \coloneqq (\vec{c}'_0, c'_1) \in \enc(\pk, x')$. 
        Consider $\ctxt'' \coloneqq (\vec{c}_0 + \vec{c}'_0 \bmod q, c_1 + c'_1 \bmod q)$.
        Derive a lower bound $\underline{q}(m,p,\beta)$ of $q$ so that $\dec(\sk, \ctxt'') = x + x'$ whenever $x + x' \in \ZZ_p$ and $q > \underline{q}(m,p,\beta)$.
        \item Generalising, let $\ell \in \NN$, 
        $(\pk, \sk) \in \kgen(\secparam)$, 
        $\vec{a} \in \ZZ_p^\ell$, $\vec{x} \in \ZZ_p^\ell$, and $\ctxt_i \in \enc(\pk, x_i)$ for all $i \in [\ell]$.
        Consider $\ctxt \coloneqq \sum_{i=1}^\ell a_i \cdot \ctxt_i \bmod q$.
        Derive a lower bound $\underline{q}'(\ell,m,p,\beta)$ of $q$ so that $\dec(\sk, \ctxt'') = \langle \vec{a}, \vec{x} \rangle$ whenever $\langle \vec{a}, \vec{x} \rangle \in \ZZ_p$ and $q > \underline{q}'(\ell,m,p,\beta)$.
    \end{enumerate} 

    
\end{enumerate}
\end{question}

\inputsol{dual_regev_homo}

\begin{question}[Normal-Form of LWE, Lindner-Peikert Encryption]
    In this question, we study the ``normal form'' of the LWE assumption and use it to prove the security of the Lindner-Peikert encryption scheme. We first recall the ordinary LWE assumption and then state the normal-form variant.
    \begin{definition*}[Decision-Learning with Errors (LWE) Assumption]
        Let $n, m, \log q \in \poly$ with $n \leq m$ and $\chi$ be a distribution over $\ZZ$ parametrised by $\secpar$.
        The Decision-$\LWE_{n,m,q,\chi}$ assumption states that for any PPT adversary $\adv$
        \[
            \abs{
            \condprob{
                b = 1
            }
            {
                \begin{aligned}
                    &\mat{A} \sample \ZZ_q^{n \times m} \\
                    &\vec{s} \sample \ZZ_q^n,~\vec{e} \sample \chi^m \\
                    &\vec{b}^\transpose \coloneqq \vec{s}^\transpose \cdot \mat{A} + \vec{e}^\transpose \bmod q \\
                    &b \gets \adv(\mat{A},\vec{b}) \\
                \end{aligned}
            }
            -
            \condprob{
                b = 1
            }
            {
                \begin{aligned}
                    &\mat{A} \sample \ZZ_q^{n \times m} \\
                    &\vec{b} \sample \ZZ_q^m \\
                    &b \gets \adv(\mat{A},\vec{b}) \\
                \end{aligned}
            }}
            \leq \negl.
        \]
    \end{definition*}

    \begin{definition*}[Normal-Form Decision-Learning with Errors (LWE) Assumption]
        Let $n, m, \log q \in \poly$ with $n \leq m$ and $\chi$ be a distribution over $\ZZ$ parametrised by $\secpar$.
        The Normal-Form Decision-$\LWE_{n,m,q,\chi}$ assumption states that for any PPT adversary $\adv$
        \[
            \abs{
            \condprob{
                b = 1
            }
            {
                \begin{aligned}
                    &\mat{A} \sample \ZZ_q^{n \times m} \\
                    &\vec{s} \sample \chi^n,~\vec{e} \sample \chi^m \\
                    &\vec{b}^\transpose \coloneqq \vec{s}^\transpose \cdot \mat{A} + \vec{e}^\transpose \bmod q \\
                    &b \gets \adv(\mat{A},\vec{b}) \\
                \end{aligned}
            }
            -
            \condprob{
                b = 1
            }
            {
                \begin{aligned}
                    &\mat{A} \sample \ZZ_q^{n \times m} \\
                    &\vec{b} \sample \ZZ_q^m \\
                    &b \gets \adv(\mat{A},\vec{b}) \\
                \end{aligned}
            }}
            \leq \negl.
        \]
    \end{definition*}
    Note that in the normal-form variant the LWE secret $\vec{s}$ is also drawn from the error distribution $\chi$. 

    Next, let $n, \log p, \log q \in \poly$ with $p < q$, and $\chi$ be the uniform distribution over $\ZZ_\beta$, for some $\log \beta \in \poly$ with $\beta$ being odd and $\beta < q$. We introduce the Lindner-Peikert encryption scheme:

    \begin{pchstack}[center,boxed]
        
        \procedure[]{$\kgen(\secparam)$}
        {
            \mat{A} \sample \ZZ_q^{n \times n} \\
            \vec{s}, \vec{e} \sample \chi^n \\
            \vec{b}^\transpose \coloneqq \vec{s}^\transpose \cdot \mat{A} + \vec{e}^\transpose \bmod q \\
            \pk \coloneqq (\mat{A}, \vec{b}) \\
            \sk \coloneqq \vec{s} \\
            \pcreturn (\pk, \sk)
        }

        \pchspace

        \procedure[]{$\enc(\pk, x \in \ZZ_p)$}
        { 	\mathsf{Parse}\;(\mat{A}, \vec{v}) \gets\pk\\
            \vec{r}, \vec{e}_0 \sample \chi^{n} \\
            e_1 \sample \chi \\
            \vec{c}_0 \coloneqq \mat{A} \cdot \vec{r} + \vec{e}_0 \bmod q \\
            c_1 \coloneqq \vec{b}^\transpose \cdot \vec{r} + e_1 + \floor{\frac{q}{p}} \cdot x \bmod q \\
            \ctxt \coloneqq (\vec{c}_0, c_1) \\
            \pcreturn \ctxt
        }

        \pchspace

        \procedure[]{$\dec(\sk, \ctxt)$}
        {
				    \mathsf{Parse}\;(\vec{c}_0, c_1) \gets\ctxt\\
				    \vec{s}\gets\sk\\
            \bar{x} \coloneqq c_1 - \vec{s}^\transpose \cdot \vec{c}_0 \bmod q \\
            \pcreturn \left \lfloor \frac{p}{q} \cdot \bar{x} \right \rceil \\
            \pccomment{rounding to nearest integer} 
        }

    \end{pchstack}

    \textbf{Choose between answering either Part (a), or answering the two Parts (b) and (c).}
    \begin{enumerate}[label=(\alph*)]
        \item Let $q$ be prime, $m \geq n + \secpar$, and $\chi$ be symmetric about $0$, i.e. $\chi = -\chi$. 
        Prove via a reduction that if the Decision-$\LWE_{n,m,q,\chi}$ assumption holds then the Normal-Form Decision-$\LWE_{n,m-n,q,\chi}$ assumption holds.
        [Hint: The analysis of normal-form SIS in the lecture notes. The level of detail of the answer should be on a similar level as the lecture notes.]
        \item Show that the Lindner-Peikert encryption scheme is correct when $q > 2 \cdot n \cdot p \cdot \beta^2$ and $n \cdot \beta^2 \geq \beta + 1$.
        [Hint: Read hint of Question 1 (b) (i)]
        \item Prove via a reduction that the Lindner-Peikert encryption scheme is IND-CPA-secure under the Normal-Form Decision-$\LWE_{n,n+1,q,\chi}$ assumption.
        [Hint: Read hint of Question 1 (b) (ii). The level of detail of the answer should be on a similar level as the lecture notes.]
    \end{enumerate}
\end{question}

\inputsol{lindner_peikert}

\begin{question}[SIS Commitments]
    In this question, we study a basic lattice-based commitment scheme. First, we introduce the concept of commitments.
    \begin{definition*}[Commitments]
        A commitment scheme for message space $\mathcal{X}$ is a tuple of PPT algorithms $\Gamma = (\mathsf{Setup}, \mathsf{Com})$ with the following syntax:
        \begin{itemize}
            \item $\pubparam \gets \mathsf{Setup}(\secparam, 1^\ell)$: The setup algorithm inputs the security parameter $\secpar \in \NN$ and a length parameter $\ell \in \NN$. It outputs the public parameters $\pubparam$ (also known as the commitment key). 
            \item $\mathsf{com} \gets \mathsf{Com}(\pubparam, \vec{x} \in \mathcal{X}^\ell; r)$: The commitment algorithm inputs the public parameters $\pubparam$, a message $\vec{x} \in \mathcal{X}^\ell$, and some randomness $r$ (from some randomness space). It outputs a commitment $\mathsf{com}$. By default, the randomness $r$ is assumed to be sampled uniformly at random from the randomness space, and is omitted from the input.
        \end{itemize}
        A commitment scheme could satisfy the hiding and binding properties defined as follows:
        \begin{description}
            \item[(Statistically) Hiding] For any $\ell \in \NN$, any $\vec{x}, \vec{y} \in \mathcal{X}^\ell$, the statistical distance between the following distributions are negligible in $\secpar$:
            \begin{align*}
                \set{
                    (\pubparam, \mathsf{com}): 
                    \begin{aligned}
                        &\pubparam \gets \mathsf{Setup}(\secparam, 1^\ell) \\
                        &\mathsf{com} \gets \mathsf{Com}(\pubparam, \vec{x})
                    \end{aligned}
                }
                && \text{and} &&
                \set{
                    (\pubparam, \mathsf{com}): 
                    \begin{aligned}
                        &\pubparam \gets \mathsf{Setup}(\secparam, 1^\ell) \\
                        &\mathsf{com} \gets \mathsf{Com}(\pubparam, \vec{y})
                    \end{aligned}
                }.
            \end{align*}
            \item[(Computationally) Binding] For any $\ell \in \NN$ and any PPT adversary $\adv$, it holds that
            \begin{align*}
                \condprob{
                    \begin{aligned}
                        &\mathsf{Com}(\pubparam, \vec{x}; r) = \mathsf{Com}(\pubparam, \vec{y}; s) \\
                        \land~&\vec{x} \neq \vec{y}
                    \end{aligned}
                }
                {
                    \begin{aligned}
                        &\pubparam \gets \mathsf{Setup}(\secparam, 1^\ell) \\
                        &((\vec{x}, r), (\vec{y}, s)) \gets \adv(\pubparam) 
                    \end{aligned}
                }
                \leq \negl.
            \end{align*}
        \end{description}
    \end{definition*}
    Let $n,m, \log p, \log q = \poly$ with $p < q$. Consider the following commitment scheme construction for the message space $\ZZ_p$:
    \begin{pchstack}[center,boxed]
    
        \procedure[]{$\mathsf{Setup}(\secparam, 1^\ell)$}
        {
            \mat{A} \sample \ZZ_q^{n \times m} \\
            \mat{B} \sample \ZZ_q^{n \times \ell} \\
            \pubparam \coloneqq (\mat{A}, \mat{B}) \\
            \pcreturn \pubparam
        }

        \pchspace

        \procedure[]{$\mathsf{Com}(\pubparam, \vec{x} \in \ZZ_p^\ell; \vec{r} \in \ZZ_p^m)$}
        {
            \vec{c} \coloneqq \mat{A} \cdot \vec{r} + \mat{B} \cdot \vec{x} \bmod q \\
            \mathsf{com} \coloneqq \vec{c} \\ 
            \pcreturn \mathsf{com}
        }

    \end{pchstack}
    \begin{enumerate}[label=(\alph*)]
        \item Prove that the above commitment scheme is statistically hiding if $m > n \cdot \log_p q + \omega(\log n)$. The level of detail of the answer should be on a similar level as the lecture notes.
        \item Prove that the above commitment scheme is computationally binding under the $\SIS_{n,m+\ell,p,q}$ assumption. The level of detail of the answer should be on a similar level as the lecture notes.
    \end{enumerate}
\end{question}

\inputsol{sis_com}

\end{document}
